% chapter1.tex -- Chapter 1: Introduction
%
% Natural Rubber Latex Thesis

\chapter{Introduction}
\label{ch:introduction}

\section{Biopolymers and Sustainable Materials}
\label{sec:biopolymers}

% Each sentence is on its own line for easier editing.

Biopolymers aren't one thing, and the labels matter because they imply different end-of-life routes.
``Biopolymer'' can mean (i) \emph{a polymer made by biology} (cellulose, starch, proteins, natural rubber)
 or (ii) \emph{a polymer made from bio-based feedstocks} (industry often uses ``biopolymer/bioplastic'' loosely).
Bio-based (renewable) means the carbon comes from contemporary biomass rather than fossil carbon;
it says nothing by itself about degradability.
Biodegradable means microbes can mineralize the material (to CO$_2$/CH$_4$, water, and biomass) under specific conditions;
the same polymer can behave very differently in soil, seawater, and industrial composting.
Compostable is a stricter subset of biodegradable: it must meet defined performance standards in composting conditions.
Circular is not a synonym for biodegradable; it is a systems goal:
keep carbon and value in circulation via reuse, mechanical/chemical recycling,
and use biodegradation only when it is a controlled, appropriate end-of-life pathway (e.g., contaminated organics packaging).

In practice, materials like \ac{PLA} are bio-based and (industrially) biodegradable;
\ac{PHA} are bio-based and biodegradable;
cellulose is a true biopolymer and biodegradable;
bio polyethylene ``bio-PE'' is renewable but \emph{not} biodegradable (it's chemically the same as PE).

This matters because regulation and brand risk are now shaping materials selection, especially in packaging.
Market tracking consistently shows packaging as the dominant biopolymer application ($\sim$38.6\% revenue share in 2023)
and biodegradable polyesters as the leading product family ($\sim$45\% share in 2023),
which aligns with the ``short-lived, high-volume'' pain point where end-of-life is the bottleneck.
Policy is a major driver of demand:
the EU's single-use plastics framework targets common disposable items (e.g., cutlery, plates, straws)
and requires substitution and redesign.
At the same time, biobased plastics remain a small slice of total plastics output
(production capacity on the order of $\sim$0.5\% of global plastics),
which is why ``biopolymers everywhere'' is not a near-term default.
Natural rubber is a renewable elastomer already produced at an industrial scale
($\sim$13.6 million tons in 2019, comparable to $\sim$15.1 million tons of synthetic rubber),
demonstrating that bio-sourced polymers can reach commodity volumes when supply chains, processing, and performance are locked in.

So, what's stopping broader adoption is mostly cost + processing + end-of-life fit, not a lack of ``green intent.''
Many bioplastics still carry a cost premium, often reported at $\sim$2--3$\times$ that of conventional plastics,
and sometimes higher in specific applications because petrochemical polymers have brutal economies of scale, optimized assets, and mature logistics.
On the processing side, biopolymers often have narrower processing windows:
moisture-driven hydrolysis (e.g., polyesters like \ac{PLA}/PBS), lower thermal stability, different crystallization kinetics (affecting shrinkage/warpage),
and rheology issues such as low melt strength or inconsistent shear-thinning, which complicate extrusion, thermoforming, and high-speed packaging lines.
Those gaps get ``patched'' with compatibilizers, chain extenders, nucleating agents, plasticizers, and multilayer structures,
with each addition increasing cost and recycling complexity.
End-of-life is the third wall:
mechanical recycling streams are sensitive to contamination;
composting requires correct collection infrastructure and clear labeling;
and ``biodegradable'' claims can backfire when disposal conditions don't match the material's biodegradation pathway.


\section{Developing Circular Economies}
\label{sec:circular-economies}

The concept of a circular economy must be understood as a rigorous materials-flow strategy rather than a synonym for ``bio-based'' or ``biodegradable'' labeling.
This distinction establishes a dialectic between the noumenon, the abstract ideal of ``circularity,'' and the phenomenon, which constitutes the measurable realities of mass balances, energy use, emissions, and waste leakage.
While the label serves as a guiding principle, it only earns its validity when the physical flows successfully close without generating excessive externalities.
The resolution of this tension requires moving beyond the philosophical adoption of sustainability terms to a quantitative audit of material lifecycles, recognizing that a loop is only truly closed when the energy and material inputs of recovery are lower than those of virgin extraction.

This thermodynamic reality challenges the circularity of fossil-derived elastomers.
The lifecycle of these materials begins with an established energy penalty involving drilling, refining, and steam cracking to produce monomers like butadiene and styrene,
and recycling aims to mitigate this by reducing virgin feedstock demand.
However, this proposition faces a stark antithesis: recycling is rarely thermodynamically free.
Mechanical recycling is constrained by property drift and contamination,
while chemical routes such as solvolysis or pyrolysis require high temperatures and purification trains that reintroduce significant heat and reagent intensity.
Consequently, the circularity of fossil polymers often amounts to trading a feedstock debt for an energy debt.

\ac{NRL} introduces a unique biological advantage to this equation, yet it faces its own material dialectic.
The thesis of \ac{NRL} is its superior carbon source: it is ``manufactured'' \emph{in planta} via photosynthesis,
sequestering carbon and arriving at the factory gate as a processable colloid without the energy intensity of synthetic polymerization.
However, the antithesis arises at the end-of-life: while raw rubber is biodegradable,
high-performance applications require vulcanization, creating a crosslinked network that resists biological breakdown and complicates reprocessing.
This physical constraint limits current circular options to downcycling (crumb rubber) or energy recovery,
rather than true biological reintegration.
The resolution of this dialectic clarifies the technical path forward: while \ac{NRL} solves the sustainable sourcing problem,
the industry must still confront the physics of the crosslinked network.


\section{Meeting Global Demand and Feedstock Diversification}
\label{sec:global-demand}

Natural rubber remains hard to replace because its \emph{cis}-1,4 polyisoprene chains undergo strain-induced crystallization,
which gives a combination of green strength, fatigue resistance, resilience, and low heat buildup that many synthetic elastomers match only partially or at higher formulation complexity.
This is why natural rubber shows up in tires, belts, hoses, vibration isolators, footwear, and medical goods,
and why tire manufacturing dominates demand; in the European Union, roughly three-quarters of natural rubber consumption is used for tires.

Global production illustrates both maturity and constraint: in 2019, natural rubber production was about 13.6 million tonnes,
approaching the roughly 15.1 million tons produced from synthetic rubber sources,
and around 85\% of natural rubber supply is linked to smallholder farming systems.
These are strengths for resilience and livelihoods, but they also create bottlenecks for quality consistency, supply stability, and rapid capacity expansion,
especially when weather, disease pressure, and price volatility intersect with rising demand from transportation and infrastructure.

Feedstock diversification is the practical hedge against those bottlenecks.
\emph{Hevea} is geographically constrained to tropical regions and has biological vulnerabilities,
so alternative latex crops such as guayule (\emph{Parthenium argentatum}) and rubber dandelion (\emph{Taraxacum koksaghyz}) are being developed as temperate-zone sources that can enable domestic supply chains and reduce exposure to regional shocks.
These crops also open application niches, including hypoallergenic latex options,
while advances in extraction and purification using tailored flocculants, chelators, and process control are improving yield and consistency.


\section{Emerging Technological Interest}
\label{sec:emerging-tech}

Natural materials often exhibit extraordinary performance because their hierarchical structures and compositions were honed through evolution.
In the specific case of elastomeric biopolymers, \ac{NRL} serves as a prime example of this complexity,
containing approximately 94\% \emph{cis}-1,4-polyisoprene and 6\% non-rubber components such as proteins and lipids.
Historically, industrial standardization viewed these biological residues as impurities to be removed,
yet they are actually critical engineering variables that link polymer chains to form a naturally reinforced network.
These components endow the material with ``green strength'' and the ability to strain-crystallize,
properties that synthetic analogues struggle to replicate.

Progressive processing is where the circular story either becomes real or stays a slogan.
In elastomer processing, mastication is the classic ``make it processable'' step,
and it works by mechanically cleaving chains to reduce viscosity.
The cost is irreversible chain scission that throws away the very long-chain physics that gives natural rubber its fatigue resistance and crack tolerance.
Quantitatively, natural rubber from latex has a number-average molecular weight around 300~kg/mol,
corresponding to roughly 4,400 repeat units per chain, while mastication can degrade chains to roughly 440 repeat units per chain,
which then requires higher crosslink densities to obtain usable networks.
That is the mastication paradox: you gain flow, but you pay with shorter strands, lower damage tolerance, and less ``room'' for \ac{SIC} to do its job.

The successful preservation of these intrinsic molecular properties directly unlocks advanced capabilities in \ac{AM}.
A major limitation in printing high-performance elastomers is the ``operational viscosity paradox,'' where the long polymer chains needed for strength create a resin too thick to print.
However, by utilizing the preserved colloidal structure of \ac{NRL}, it is possible to maintain ultra-high molecular weights within discrete particles while keeping the bulk viscosity low, akin to flowing water.
This synergy between living feedstocks and robotic control grants engineers an unprecedented degree of freedom to design responsive, hierarchical materials.


\section{Problem Statement and Research Motivation}
\label{sec:problem-statement}

\subsection{Preservation Chemistry}
\label{subsec:preservation-chemistry}

Fresh natural rubber latex is a reactive biological colloid, so it will not stay ``liquid and usable'' on its own.
After tapping, microbial growth and enzyme-driven chemistry shift the serum conditions, destabilize the rubber particle interface, and trigger spontaneous coagulation and putrefaction, which makes the latex malodorous and unprocessable.
Preservation is therefore a logistics requirement, not an optional additive, because it must keep latex stable during storage, transport, and downstream conversion.

The historical solution is alkaline preservation, where ammonia raises pH to suppress microbial activity while also stabilizing rubber particles through electrostatic repulsion, often with secondary stabilizers such as zinc oxide and TMTD.
The engineering and environmental cost of this approach is now hard to ignore: ammonia is volatile and hazardous to handle, it drives wastewater constraints that smallholders struggle to manage, and its high alkalinity can discolor latex, corrode equipment, and add downstream neutralization burden.
Meanwhile, TMTD can generate carcinogenic nitrosamines under high-temperature processing.

An ideal preservative for \ac{NRL}, therefore, has a clear technical job description:
\begin{itemize}
    \item Inhibit microbial growth strongly enough to prevent acidification and putrefaction
    \item Maintain colloidal stability by increasing surface charge and electrokinetic potential
    \item Control trace multivalent ions through sequestration or precipitation
    \item Be non-volatile, low-toxicity, and easy to handle
    \item Avoid discoloration and odor
    \item Minimize corrosion and effluent burden
    \item Remain compatible with established concentration and coagulation steps
    \item Preserve the native protein--lipid interphase and molecular integrity
\end{itemize}


\subsection{Processing Challenges: The Gap Between Microstructure and Macroscopic Flow}
\label{subsec:processing-challenges}

The central motivation for this research is the critical need to bridge the disconnect between the microscopic properties of \ac{NRL} and its macroscopic processability.
\ac{NRL} is not a simple fluid, but a sophisticated colloidal suspension of polyisoprene particles modified by a complex interface of proteins and lipids.
Key processing parameters, including viscosity, yield stress, thixotropy, and post-shear recovery, are intrinsically governed by the arrangement and interaction of these particles.
However, to engineer high-performance adhesives or 3D printing inks, we must move beyond empirical observations to predictive models that correlate these structural characteristics with rheological behavior.

Current theoretical frameworks are insufficient for this task.
Existing models for complex sphere suspensions describe how relative viscosity diverges near maximum packing and explain shear thickening via hydrodynamic clustering, but they fail to address the unique complexities of deformable, core-shell rubber particles.
At high volume fractions, hydrodynamic lubrication becomes dominant, requiring models that incorporate both colloidal forces (hydrodynamic drag, Brownian motion, electrostatic repulsion) and the elastic properties of the polymer core.


\subsection{Deficiency of Models and the Need for Standardization}
\label{subsec:model-deficiency}

This theoretical gap is compounded by practical processing constraints and a historical bias in the available data.
Industrial processing is increasingly constrained by stringent regulations, such as the OSHA 8-hour ammonia emission limit (50~ppm), which necessitates a shift toward alternative preservation chemistries.
However, switching preservatives fundamentally alters the physical properties of the latex, including surface charge density, protein conformation, and mechanical and thermal behavior.
Current rheological models are ``conservative'' in that they are calibrated almost exclusively on ammoniated \emph{Hevea} latex or synthetic systems, meaning our foundational knowledge is biased toward a material standard that is becoming obsolete.

Furthermore, these models overlook critical biological variations, such as the distinct particle size distributions observed in alternative species: Guayule (0.44--2~$\mu$m) and Dandelion (0.35~$\mu$m),
which differ significantly from those of \emph{Hevea}.
To address this, we must benchmark \ac{NRL} across different conditions to build a unified classification framework based on:
\begin{enumerate}
    \item[(I)] \textbf{Origin}: Hevea, Guayule, Dandelion
    \item[(II)] \textbf{Preservation System}: High/Low Ammonia, TMTD, Acid-Surfactant
    \item[(III)] \textbf{Quantifiable Properties}: Particle Size Distribution, Solids Content, Molecular Weight
\end{enumerate}


\section{Research Objectives and Scope}
\label{sec:objectives}

\subsection{Core Problem}
\label{subsec:core-problem}

Natural rubber latex is processed and modeled largely through the lens of ammonia-preserved systems.
As the field moves toward safer, low-to-zero ammonia preservation, latex behaves like a different material class because preservation chemistry alters the particle interface, non-rubber constituents, and network formation.
The result is a gap: we do not yet have a unified structure-rheology-processability framework that spans preservation conditions, so engineering choices become trial-and-error, and advanced manufacturing routes remain underexploited.

\subsection{Research Objective and Scope}
\label{subsec:research-scope}

This thesis builds a preservation-aware framework that links measurable microstructure and chemistry to rheology and manufacturability, then uses that understanding to enable sustainable processing routes, including photocurable latex systems for additive manufacturing.

\begin{description}
    \item[Objective 1:] Map how preservation chemistry reshapes flow and microstructure
    \item[Objective 2:] Identify preservation-dependent chemical and dynamical signatures using high-sensitivity NMR
    \item[Objective 3:] Translate preserved latex into manufacturable, photocurable feedstocks for additive manufacturing
\end{description}


\section{Thesis Organization}
\label{sec:organization}

\textbf{Chapter~\ref{ch:introduction}} (this chapter) introduces the background on biopolymers, circular economy concepts, natural rubber latex fundamentals, and the research objectives.

\textbf{Chapter~\ref{ch:literature-review}} presents a comprehensive literature review covering molecular structure and colloidal stabilization of \ac{NRL}, suspension rheology and theoretical models, and additive manufacturing of elastomers.

\textbf{Chapter~\ref{ch:methodology}} describes the research methodology, including materials sourcing, rheological characterization protocols, NMR spectroscopy methods, and photoresin formulation and testing procedures.
