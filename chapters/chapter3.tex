% chapter3.tex -- Chapter 3: Research Methodology
%
% Natural Rubber Latex Thesis

\chapter{Research Methodology}
\label{ch:methodology}

\section{Sourcing, Traceability, and Rationale: Natural Rubber Latex Preservation Systems}
\label{sec:sourcing}

% Each sentence is on its own line for easier editing.

\subsection{Overview of Preservation Methods as the Master Variable}
\label{subsec:preservation-overview}

The fundamental research design rests upon systematic comparison of two distinct preservation methodologies for \ac{NRL}, each representing different approaches to maintaining polymer stability and functional properties.
This comparison serves as the Master Variable throughout the experimental program, directly addressing the research objective to define how preservation chemistry alters \ac{NRL} microstructure and flow characteristics.
The primary motivation for this comparison derives from documented preservation and processing challenges in \ac{NRL} production, particularly concerning ammonia toxicity, environmental sustainability, and the need to maintain microbial stability without hazardous volatiles.


\subsection{Ammoniated Latex System}
\label{subsec:ammoniated-latex}

AFLatex Technologies LDA (Victoria, Caldas, Colombia) provided the ammoniated \ac{NRL} sample, representing the historical preservation standard.
This ammoniated latex was supplied as a centrifuged system with 60\% \ac{DRC} according to ASTM D1076-15 classification standards.
The ammoniated formulation employs ammonia as the primary preservative and stabilizer, which has been the traditional approach for \ac{NRL} conservation for decades.
However, this material serves a dual role in the research: it functions both as a baseline against which eco-preserved systems are evaluated and as a means to investigate ammonia's documented effects on protein and phospholipid retention within the latex colloidal system.


\subsection{Eco-preserved Latex Systems}
\label{subsec:eco-preserved}

AFLatex Technologies LDA provided two distinct eco-preserved \ac{NRL} formulations, both ammonia-free, which form the core of the comparative preservation study:
\begin{itemize}
    \item The first system, designated \textbf{Alfa}, employs a preservation chemistry combining ethoxylated tridecyl alcohol and hydrofluoric acid.
    \item The second system, designated \textbf{Beta}, utilizes linear dodecylbenzene sulfonic acid as the primary preservative.
\end{itemize}

Both materials were supplied in two distinct solid content formulations, approximately 60\% and 30\% \ac{DRC}, allowing investigation across a range of particle volume fractions while maintaining identical preservation chemistry.
Additional latex serum was provided to prepare intermediate concentrations as needed.


\subsection{Reference and Synthetic Polymer Materials}
\label{subsec:reference-materials}

To contextualize the natural rubber latex system within the broader landscape of elastomeric polymers, two reference materials were included:
\begin{itemize}
    \item Synthetic polyisoprene (L-IR-50) with an average molecular weight of 54,000~Da was donated by Kuraray Co. Ltd. (Tokyo, Japan).
    \item Deproteinized and saponified liquid natural rubber (DPR-40) with an average molecular weight of 40,000~Da was donated by DPR Industries, a division of Pacer Industries Inc. (Coatesville, PA, USA).
\end{itemize}

These materials allow distinction between the rheological and NMR signatures attributable to preservation chemistry versus those arising from the native protein-lipid matrix inherent to natural rubber latex.


\subsection{Pre-receipt Specifications}
\label{subsec:pre-receipt}

The receipt-stage characterization was used to benchmark incoming natural rubber latex against ISO 2004:2010 and related procedures (e.g., ASTM D1076 for \ac{DRC}).
The objective is lot-level traceability and verification of baseline quality prior to formulation and rheological testing.
Key standardized indicators include:
\begin{itemize}
    \item Mechanical stability time (MST)
    \item Volatile fatty acid number/index (VFA)
    \item Solids content (TSC/\ac{DRC})
    \item Brookfield viscosity
\end{itemize}


\subsection{Incoming Verification}
\label{subsec:incoming-verification}

Upon receipt, each latex batch was independently verified to confirm it matched the supplier's CoA and met baseline quality requirements.
All results were recorded on an Incoming Verification Sheet linked to supplier, lot/batch code, ship/storage conditions, and the intended use condition; batches failing criteria were quarantined or rejected.

\ac{DRC} was measured by ASTM D1076: $\sim$10~g latex was diluted to $\sim$25~wt\% total solids, coagulated with 2~wt\% acetic acid under stirring, washed/rolled, then dried at 70°C (or 55°C if oxidation was observed) to constant mass.
Batches were accepted only when measured \ac{DRC} agreed with the supplier value within $\pm$1~wt\% (absolute).

Particle-level stability was checked by \ac{DLS}/zeta potential using a 1:100--1:1000 dilution in a defined ionic medium (10~mM electrolyte) to reduce double-layer artifacts and multiple scattering; $Z$-average diameter, \ac{PDI}, and zeta potential were reported with dilution factor and diluent composition.


\section{Spectroscopic and Analytical Reagents}
\label{sec:reagents}

\subsection{Deuterated Solvents}
\label{subsec:deuterated-solvents}

Deuterium oxide (D$_2$O, CAS: 7789-20-0) and deuterated chloroform (CDCl$_3$, CAS: 865-49-6, 99.8 atom\% deuteration, containing 0.03\% tetramethylsilane [TMS] as internal standard) were purchased from Merck (Sigma-Aldrich).
Both solvents were stored over 4~Å molecular sieves to maintain isotopic purity and prevent isotopic exchange with atmospheric moisture.


\subsection{Photopolymerization and Surfactant Reagents}
\label{subsec:photo-reagents}

\begin{itemize}
    \item Trimethylolpropane triacrylate (\ac{TMPTA}, technical grade, 246808)
    \item Phenylbis(2,4,6-trimethylbenzoyl)-phosphine oxide (\ac{TPO}, 97\%, 511447)
    \item Tartrazine (Acid Yellow 23, $\geq$85\%, T0388)
    \item Tween 20 (polyoxyethylene (20) sorbitan monolaurate)
\end{itemize}
were obtained from Millipore Sigma, USA.

\begin{itemize}
    \item Ebecryl 114 (2-phenoxyethyl acrylate, 024572A)
    \item Ebecryl IBOA (isobornyl acrylate monomer, 024944B01Z01)
    \item Ebecryl 8413 (pigment-grind urethane-acrylate oligomer, 029588A)
\end{itemize}
were supplied by Allnex, USA.

1,6-hexanediol diacrylate (\ac{HDDA}, 99\%, 043203.30) was purchased from Thermo Scientific, USA. Span 80 (sorbitan monooleate, S0060) was ordered from TCI Chemicals. Acetate buffer solution (pH 4.66, 1.07827.1000) was acquired from Merck, Germany.


\section{Rheological Characterization}
\label{sec:rheology}

Understanding how \ac{NRL} flows and deforms under stress is critical because flow behavior reflects the microstructure and dictates processability in additive manufacturing.
This section details the rheological protocols, the colloidal phenomena being probed, and the models used to interpret the data.


\subsection{Sample Preparation and Volume-Fraction Series}
\label{subsec:sample-prep}

Rheology was performed on latex suspensions prepared over a solids volume-fraction range of $\phi = 0.2$--$0.6$ to isolate the effect of concentration on flow and viscoelastic response.
For the ammonia-free latex systems, the stock latex at the highest concentration was diluted to target $\phi$ using the matching latex serum to preserve the native ionic environment.
For ammonia-preserved latex, dilution was performed using deionized water rather than ammonia serum due to handling hazards; pH and conductivity were recorded for each diluted sample.


\subsection{Rotational Rheometry (Parallel Plates)}
\label{subsec:rotational-rheometry}

Steady and oscillatory measurements were conducted using a NETZSCH Kinexus rotational rheometer with a 40~mm upper / 60~mm lower parallel-plate configuration and a 0.5~mm gap at room temperature.
Approximately 1~mL of latex was loaded for each test.

\begin{enumerate}[label=(\roman*)]
    \item \textbf{Steady shear ramps} were performed from 0.01~s$^{-1}$ to 300~s$^{-1}$ to obtain viscosity--shear rate curves and identify Newtonian plateaus and shear-thinning regions.
    Where thixotropy was relevant, an up--down ramp (0.01 $\rightarrow$ 300 $\rightarrow$ 0.01~s$^{-1}$) was used to assess hysteresis.
    
    \item \textbf{Oscillatory amplitude sweeps} were used to determine the linear viscoelastic window and quantify storage and loss moduli ($G'$, $G''$) as a function of strain amplitude.
    Sweeps were collected over 0.01--200\% strain at fixed frequencies of 0.1~Hz, 1~Hz, and 10~Hz.
    
    \item \textbf{Shear start/recovery (thixotropy) tests} were conducted using a three-interval thixotropy test (3ITT): a low-shear interval to establish a reference state, a high-shear interval to induce structural breakdown, and a final low-shear interval to quantify recovery.
\end{enumerate}


\subsection{Computational Fluid Dynamics}
\label{subsec:cfd}

Taylor--Couette measurements and simplified two-phase-flow interpretation.
At higher volume fractions where parallel-plate testing can be affected by wall slip and particle migration, a Taylor--Couette (concentric cylinder) geometry was used to provide more reliable high-$\phi$ measurements.
The inner cylinder radius was 0.64~cm, and the outer cylinder radius was 2.54~cm; the inner cylinder rotated at a fixed 55~rpm while the outer cylinder remained stationary.
To interpret concentration nonuniformity and migration trends observed in Couette flow, a simplified two-phase (suspension balance) framework was used: the mixture flow field is solved with no-slip at the walls, and the particle phase is allowed to redistribute via a concentration-transport equation that captures shear-induced migration and buoyancy/settling effects.
This modeling was used as an interpretive tool to confirm that observed viscosity changes with $\phi$ are consistent with expected migration/stability behavior in Couette flow and to support attributing trends to intrinsic particle/serum effects rather than measurement artifacts.


\subsection{Rheology Interpretation Framework}
\label{subsec:rheology-framework}

To interpret viscosity--shear rate data across volume fraction $\phi$, a microstructure-based picture is used in which latex particles form transient linkages (flocs/bridges) at rest and under low shear, and these linkages are progressively disrupted under increasing shear.

The internal structural state is represented by $N$, the instantaneous number of effective interparticle linkages contributing to resistance to flow, and $N_0$, the maximum linkage density at rest.
Under shear, linkages break at a rate that scales with both how much structure exists and how strong the imposed deformation is:
\begin{equation}
    \frac{dN}{dt} = -k_d N \dot{\gamma}^m + k_r(N_0 - N)
    \label{eq:linkage-kinetics}
\end{equation}
where $k_d$ is the breakage rate constant, $\dot{\gamma}$ is shear rate, $m$ is the shear-sensitivity exponent, and $k_r$ is the reformation rate constant.

At steady state ($dN/dt = 0$), the normalized structure becomes:
\begin{equation}
    \frac{N}{N_0} = \frac{1}{1 + a\dot{\gamma}^m}, \quad a = \frac{k_d}{k_r}
    \label{eq:steady-state-structure}
\end{equation}

The steady shear viscosity is then written using the Cross model:
\begin{equation}
    \eta(\dot{\gamma}) = \eta_\infty + \frac{\eta_0 - \eta_\infty}{1 + a\dot{\gamma}^m}
    \label{eq:cross-model}
\end{equation}


\subsection{Viscosity and Volume Fraction}
\label{subsec:viscosity-volume-fraction}

To relate viscosity to volume fraction $\phi$, the classical Mooney/Krieger--Dougherty framework for concentrated suspensions is used.
The Krieger--Dougherty expression predicts a divergence of $\eta_r$ as $\phi$ approaches an effective critical/maximum packing fraction $\phi_c$:
\begin{equation}
    \eta_r(\phi) = \left(1 - \frac{\phi}{\phi_c}\right)^{-[\eta]\phi_c}
    \label{eq:kd-model}
\end{equation}

To capture the experimentally observed sharper upturn near $\phi_c$ in preserved natural latex, an extended form is used:
\begin{equation}
    \boxed{\eta_r(\phi) = \left(1 - \frac{\phi}{\phi_c}\right)^{-[\eta]\phi_c} + \exp\left[a(\phi - \phi_c) + b\right]}
    \label{eq:extended-kd}
\end{equation}
where $a$ and $b$ are empirical fit parameters.
Here, $a$ controls how rapidly the additional thickening accelerates, while $b$ sets its baseline magnitude.


\subsection{Addressing Measurement Challenges and Micro-Level Effects}
\label{subsec:measurement-challenges}

A rotational rheometer does not measure viscosity directly; it measures torque $M$ and angular velocity $\Omega$, from which shear stress $\tau$, shear rate $\dot{\gamma}$, and apparent viscosity $\eta_{\text{app}}$ are inferred via geometry-dependent factors $K_\tau$ and $K_\gamma$ obtained by solving the Navier--Stokes equations under idealized conditions.
In compact form:
\begin{equation}
    \tau = K_\tau M, \quad \dot{\gamma} = K_\gamma \Omega, \quad \eta_{\text{app}} = \frac{\tau}{\dot{\gamma}} = \frac{K_\tau}{K_\gamma} \frac{M}{\Omega}
    \label{eq:rheometer-relations}
\end{equation}

These relations are only valid if three physical assumptions hold:
\begin{enumerate}[label=(\roman*)]
    \item no slip at the walls ($v_{\text{fluid}} = v_{\text{wall}}$),
    \item homogeneity across the gap ($\phi$ and hence $\eta$ independent of position),
    \item laminar simple shear without secondary flows.
\end{enumerate}

In concentrated \ac{NRL}, all three are easily violated.
Depletion layers at smooth tools create wall slip, so only a thin solvent-rich layer is actually sheared; the instrument overestimates $\dot{\gamma}$ and underestimates $\eta$.
Shear-induced migration in geometries with strong shear-rate gradients (parallel plate, wide-gap Couette) drives particles from high-shear to low-shear regions, generating a spatially varying $\phi(\mathbf{r})$ and a torque that is a nontrivial average over a heterogeneous microstructure.
At higher rotational speeds in cylindrical geometries, inertial instabilities (Taylor vortices, wavy vortices) add an extra, non-constitutive contribution to the torque, which appears as artificial shear thickening if interpreted with the simple $\eta \propto M/\Omega$ relation.

Mitigation, therefore, combines experimental design and analytical correction to keep the inferred viscosity as close as possible to the true constitutive response.
Wall slip is minimized by using serrated or sand-blasted tools to mechanically couple the bulk to the walls; when unavoidable slip remains, measurements across multiple gaps are analyzed using Mooney-type constructions to estimate slip velocity and correct the true shear rate.
Shear-induced migration is reduced at the source by choosing geometries with nearly uniform shear (small-angle cone-and-plate, narrow-gap Couette) and by limiting measurement times at high shear so that strong concentration gradients do not fully develop.
When gradients are expected, the data are interpreted within a suspension-balance or two-phase framework, where particle fluxes driven by $\nabla\dot{\gamma}$ and $\nabla\phi$ are coupled to a local viscosity $\eta(\phi)$ to rationalize deviations from ideal behavior.
Finally, flow instabilities are avoided by operating below the critical Taylor/Reynolds numbers for the chosen gap-to-radius ratio, and by favoring narrow-gap or outer-rotating Couette configurations that delay the onset of vortices.
Together, these strategies ensure that the reported \ac{NRL} viscosities and fitted constitutive parameters reflect microstructure-controlled material properties rather than artifacts of migration, slip, or inertial flow.


\section{Nuclear Magnetic Resonance (NMR)}
\label{sec:nmr}

\subsection{Overview and General Conditions}
\label{subsec:nmr-overview}

All NMR experiments were performed at 303~K on two solution-state instruments:
\begin{enumerate}[label=(\roman*)]
    \item a Bruker Avance I 600~MHz equipped with a Prodigy cryoprobe (Z150313\_001; cpT4600ss3H\&F-LIN-D-05Z)
    \item a Bruker Nyx / Bruker NEO 500~MHz fitted with a Prodigy-BBO probe (Z130036\_0001; CPP BBO 500S2BB-H\&F-D052LT)
\end{enumerate}

Chemical shifts were internally referenced using TMS ($\delta_H = 0$~ppm), CDCl$_3$ ($\delta_H = 7.26$~ppm), or D$_2$O/HDO ($\delta_H = 4.79$~ppm), selected based on the solvent system used for each measurement.


\subsection{Sample Preparation (Solution-State NMR)}
\label{subsec:nmr-sample-prep}

Unless otherwise specified, samples were prepared by dissolving natural rubber latex at 10~mg per 0.6~mL D$_2$O in standard 5~mm NMR tubes.
To capture compositional and processing variability, solution-state NMR was conducted on multiple \ac{NRL}-derived sample classes under both ammonia-preserved and ammonia-free conditions:
\begin{itemize}
    \item field latex (ammonia and ammonia-free)
    \item an industrial serum fraction from the ammonia-free system
    \item concentrated latex (ammonia and ammonia-free)
    \item ultracentrifugation-fractionated latex prepared from each preservation condition
\end{itemize}

\ac{NRL} was fractionated by ultracentrifugation at 25,000~rpm and 277~K, yielding a reproducible three-layer separation: a top cream layer (enriched in intact rubber particles), followed by two aqueous serum layers (Serum B and Serum C).


\subsection{High-Resolution Solution-State NMR (1D and 2D)}
\label{subsec:high-res-nmr}

The following Bruker pulse programs were used:
\begin{itemize}
    \item $^1$H 1D: zg30 (NS = 8)
    \item $^{13}$C 1D: zgpg30 (NS = 6400)
    \item $^1$H--$^{13}$C HSQC (multiplicity-edited): hsqcedetgpsisp2p.3 (NS = 8)
    \item HMBC: hmbcgpndqf (NS = 8)
    \item $^1$H--$^1$H COSY: cosygpprqf.uw (NS = 64)
    \item $^{31}$P 1D: zgig30 (NS = 256)
    \item DEPT-135: dept-135 (NS = 1005)
\end{itemize}

Data were processed in MestReNova. Free induction decays (FIDs) were Fourier transformed after zero-filling to 2$\times$ points.


\subsection{Diffusion-Ordered Spectroscopy (DOSY)}
\label{subsec:dosy}

\ac{DOSY} experiments used the ledbpgp2s sequence with: NS = 16, receiver gain = 3.5, relaxation delay = 2~s, pulse width = 7.07~$\mu$s, and acquisition time = 2.7739~s.
Diffusion coefficients $D$ were obtained by fitting the diffusion-dependent signal attenuation to the Stejskal--Tanner relation:
\begin{equation}
    I(g) = I_0 \exp\left[-b(g) D\right]
    \label{eq:stejskal-tanner}
\end{equation}
with
\begin{equation}
    b(g) = \gamma^2 g^2 \delta^2 \left(\Delta - \frac{\delta}{3}\right)
    \label{eq:b-factor}
\end{equation}
where $I(g)$ is peak intensity at gradient amplitude $g$, $I_0$ is intensity at $g = 0$, $\gamma$ is gyromagnetic ratio, $\delta$ is gradient pulse duration, and $\Delta$ is diffusion delay.

Hydrodynamic radius was estimated using the Stokes--Einstein equation:
\begin{equation}
    r_h = \frac{k_B T}{6\pi\eta D}
    \label{eq:hydrodynamic-radius}
\end{equation}


\subsection{Time-Domain NMR Relaxometry (TD-NMR)}
\label{subsec:td-nmr}

TD-NMR was used to probe relaxation dynamics associated with distinct molecular environments:
\begin{itemize}
    \item $T_1$ was measured using an inversion recovery sequence (t1ir, NS = 2) over 11 delay points.
    \item $T_2$ was measured using a CPMG sequence with d1 = 4~s, d20 = 0.001~s, L4 = 2, and NS = 8.
\end{itemize}

$T_1$ relaxation times were extracted by fitting to a mono-exponential recovery model:
\begin{equation}
    M(t) = M_0 \left(1 - 2e^{-t/T_1}\right) + C
    \label{eq:t1-recovery}
\end{equation}

$T_2$ relaxation was quantified from CPMG decay curves by fitting to a biexponential decay model:
\begin{equation}
    M(t) = A_1 e^{-t/T_{2,1}} + A_2 e^{-t/T_{2,2}} + C
    \label{eq:t2-decay}
\end{equation}


\section{Components and Protocols of NRL Photoresin Formulation}
\label{sec:photoresin}

Advanced manufacturing of \ac{NRL} requires a carefully engineered photoresin that balances printability, cure kinetics, and mechanical performance.


\subsection{Preparation of the UV-Curable Latex-Scaffold Coalescence}
\label{subsec:uv-curable-latex}

0.9~wt\% \ac{SDS} was added to 60\% \ac{DRC} ammonia-free \ac{NRL} and mixed for 3~min using a magnetic stir bar in a 100~mL beaker.
Subsequently, approximately 4.44~wt\% \ac{HDDA} was incorporated into the solution under low Kelvin lighting conditions with slow dispersion mixing for 5~min.
Following this, approximately 4.5~wt\% \ac{TPO} photoinitiator was added, and the mixture underwent slow 1.5-h mixing to reduce the operational viscosity further since \ac{NRL} is shear thinning.
The beaker was sealed with Saranwrap to prevent dehydration and excessive air interaction, while an aluminum foil cover was utilized to minimize light exposure.


\subsection{Preparation of Photoresin Emulsion (PRE)}
\label{subsec:pre-preparation}

High-internal-phase oil-in-water (O/W) photoresin emulsions were prepared at a 60:40 oil-to-water weight ratio:
\begin{itemize}
    \item \textbf{Oil phase} (60~wt\% of total emulsion): base monomer (\ac{HDDA} or \ac{TMPTA}) supplemented with 1.0~wt\% photoinitiator (\ac{TPO}) and 0.75~wt\% low-HLB surfactant (Span 80).
    \item \textbf{Aqueous phase} (40~wt\% of total emulsion): acetate buffer containing 4.25~wt\% high-HLB surfactant (Tween 20).
\end{itemize}

The total surfactant concentration was fixed at 5~wt\% of the final emulsion, with a Span:Tween weight ratio of 3:17 to achieve an effective HLB $\approx$ 11.

Emulsification was performed using an overhead stirrer equipped with a four-blade pitched-blade turbine (MINISTAR 20, IKA Works, Germany).
The oil and aqueous phases were pre-mixed separately and equilibrated to 20--23°C.
With the aqueous phase stirred at 800--1,200~rpm, the oil phase was added as a thin, steady stream over 60--90~s.


\subsection{Preparation of Dual Emulsion Photoresin (DEPR)}
\label{subsec:depr-preparation}

DEPRs were prepared by blending the O/W PRE with \ac{NRL}.
To evaluate the effect of photoresin loading, a Design of Experiments (DOE) approach was used, varying the PRE content to 16, 25, 32, and 44~wt\% relative to the total dual-emulsion mass, with the balance comprising the \ac{NRL} stock.
Both \ac{HDDA}- and \ac{TMPTA}-based emulsions were evaluated at these distinct loading levels.


\section{Material Characterizations}
\label{sec:characterization}

\subsection{Particle Size and Zeta Potential}
\label{subsec:particle-size}

The droplet size and surface charge of the photoresin emulsions were characterized using \ac{DLS} and electrophoretic light scattering (ELS), respectively, on a Malvern Zetasizer Nano ZSP (Malvern Panalytical, UK).
The instrument was equipped with a 10~mW He--Ne laser (632.8~nm) and operated with non-invasive backscatter (NIBS) optics at a detection angle of 173°.

Samples were prepared by diluting the \ac{NRL} or photoresin emulsions into 2-mM acetate buffer (pH 4.66) to a final concentration of 0.01--0.05~wt\% to suppress multiple scattering.
Measurements were performed in disposable folded capillary cells (DTS1070) at 25°C after a 120~s equilibration period.


\subsection{(Photo)rheology Characterizations}
\label{subsec:photorheology}

Rheological measurements were performed using a Discovery HR-20 hybrid rheometer (TA Instruments, USA) utilizing a 20~mm parallel-plate geometry.
Gap height was set to 0.5~mm for O/W photoresin emulsions.
For DEPRs, the gap was reduced to 0.2~mm to ensure uniform UV intensity across the sample depth.

\begin{enumerate}[label=\Roman*.]
    \item \textbf{Shear viscosity}: Flow behavior was characterized via steady-state shear experiments.
    Viscosity profiles were obtained by ramping the shear rate from 0.01 to 200~s$^{-1}$ at 25°C.
    
    \item \textbf{In-situ photorheology}: Real-time photopolymerization kinetics were monitored using the UV-Curing Accessory Kit (TA Instruments).
    UV irradiation was supplied by an OmniCure Series 2000 system equipped with a 200~W mercury arc lamp (320--500~nm).
    The incident intensity at the sample surface was calibrated to 30~mW~cm$^{-2}$.
    Curing profiles were recorded via oscillatory time sweeps at a fixed frequency of 5~Hz and a strain amplitude of 0.3\%.
\end{enumerate}


\subsection{Indirect Manufacturing of Tensile Testing Specimens for UV-Curable Latex-Scaffold Coalescence}
\label{subsec:indirect-manufacturing}

In the fabrication of dogbone-shaped specimens using UV-curable natural rubber latex (\ac{NRL}), a setup inspired by traditional top-down geometric configuration VAT was utilized.
Initially, injection-molded ASTM D412 Die C shapes made from high-density polyethylene were used as molds for thermoforming PET sheets.
After softening the sheets sufficiently, the dogbone shapes were formed using the injection-molded Die C molds.
A light source was prepared using an Omniture S2000 high-pressure mercury light guide, which was positioned 10~mm above the mold.
UV-curable \ac{NRL} was applied layer by layer using a 3~mL syringe, following a bottom thin layer method to minimize air bubbles in the PET mold.
Each layer was cured under UV light for 30~seconds, and samples were prepared for two intensities: 18 and 30~mW/cm$^2$.
This layering process was repeated until five layers were added to each specimen, resulting in precisely fabricated samples for mechanical and viscoelastic characterization.

After the indirect 3D printing process, the specimens underwent several treatments to enhance their mechanical integrity and stability.
First, they were soaked in isopropyl alcohol for 30~minutes to remove any residual uncured resin.
Next, they were exposed to low-intensity UV light for 10~minutes from a Black-Ray UV bench lamp (365~nm, 115~V--60~Hz) with an intensity of approximately 10--15~mW/cm$^2$ to further harden the material.
Finally, the samples were placed in an Isotemp vacuum oven (Model 282A) at 65°C and 30~mmHg for 10~hours to ensure dehydration.
Throughout this process, the weight loss of the specimens was monitored before and after processing to maintain consistency in the material properties.


\subsection{Fabrication of Jammed Microreinforced Elastomers (JMRE) and Controls}
\label{subsec:jmre-fabrication}

A two-stage curing process (UV irradiation followed by thermal treatment) was employed to transform liquid DEPR into solid, jammed, micro-reinforced elastomers (JMRE).
This nomenclature reflects the transition from a jammed micro-emulsion state to a reinforced elastomeric composite.

\begin{enumerate}[label=\Roman*.]
    \item \textbf{UV-Curing and specimen molding}: To prepare mechanical test specimens, custom molds were fabricated by casting translucent tin-cure silicone (Smooth-On, Macungie, USA) against 3D-printed masters (ASTM D638 Type V dogbone geometry for tensile tests; flat sheets for fracture/puncture tests).
    The liquid DEPR was cast into the silicone molds and exposed to UV irradiation (30~mW$^{-2}$) using an OmniCure S200 Elite system.
    This step locked the photoresin phase (PRE) into a rigid porous scaffold, establishing the green composite structure.
    
    \item \textbf{Thermal treatment and latex coalescence}: Immediately following UV curing, the specimens were demolded and transferred to a dehydrator/vacuum oven.
    This step removed residual water and induced osmotic destabilization, thereby forcing the close-packed rubber particles to coalesce within the photoresin scaffold.
    The samples were then thermally cured at 70°C overnight to ensure complete formation of the latex film.
    Demolding before thermal treatment was critical to minimize shrinkage-induced stress and prevent warping due to thermal expansion mismatches.
    
    \item \textbf{Preparation of control samples}: Control samples of pure porous photoresin were prepared by UV-curing the PRE (25~wt\% stock) under identical conditions.
    Pure natural rubber (NR) controls were prepared by casting the \ac{NRL} into molds and chemically coagulating the surface with alcohol to induce a weak gel state similar to the jammed DEPR precursor.
    These gelled NR samples were then subjected to the same 70°C thermal treatment to ensure a comparable thermal history and diffusion profile.
\end{enumerate}


\subsection{Morphological Characterization (SEM)}
\label{subsec:sem}

Surface and cross-sectional morphologies were examined using a Gemini SEM 450 (Zeiss, Germany).
The microscope was operated at an accelerating voltage of 3.00~kV with a working distance of $\sim$9.2~mm to minimize beam damage and charging on the polymeric samples.

\begin{enumerate}[label=\Roman*.]
    \item \textbf{Porous photoresin scaffolds}: Porous PR samples were mounted by pressing aluminum stubs equipped with double-sided conductive carbon tape directly onto the sample surface to preserve the native porous architecture.
    
    \item \textbf{JMRE samples}: To analyze the internal microstructure and failure mechanisms, imaging was performed on the fracture surfaces of specimens recovered after tensile testing.
    
    \item \textbf{Sputter coating}: Before imaging, samples were sputter-coated with a finer-grained Platinum coating to visualize the tiny pores without introducing artificial roughness.
    With the JMRE, given that the fracture surface is rough and macroscopic relative to pores, conventional gold was sufficient for electrical conductivity using a Leica EM ACE600 high-vacuum coater.
    To mitigate charging effects on the rough fracture surfaces, the coating thickness was optimized to 10~nm for the JMRE composites, while a 5~nm layer was applied to the porous PR samples to prevent obscuring fine pore details.
\end{enumerate}


\section{Mechanical Characterizations}
\label{sec:mechanical-testing}

Mechanical testing, including quasi-static tensile testing, cyclic fatigue, step-cyclic loading, fracture energy, and puncture resistance, was performed using a universal testing machine (Instron 5967, USA).
The system was equipped with interchangeable 50~N and 30~kN load cells.


\subsection{Uniaxial Tension Test}
\label{subsec:tensile-test}

Quasi-static tensile tests were performed on Type V dog-bone specimens at a constant crosshead speed of 500~mm~min$^{-1}$ until failure ($n = 4$--5).
Engineering stress ($\sigma$) was calculated as the measured force divided by the initial cross-sectional area, and engineering strain ($\varepsilon$) as the displacement divided by the initial gauge length.
Young's modulus ($E$) was determined from the linear slope of the stress-strain curve in the low-strain region (1--10\%).

Fracture energy density ($W_f$), representing the total energy absorption capacity, was calculated by integrating the area under the stress-strain curve:
\begin{equation}
    W_f = \int_0^{\varepsilon_m} \sigma \, d\varepsilon
    \label{eq:fracture-energy-density}
\end{equation}


\subsection{Cyclic and Hysteresis Tests}
\label{subsec:cyclic-tests}

Cyclic tests were conducted under displacement control on Type V specimens.
Samples were cycled at 100\% strain amplitude and a crosshead speed of 50~mm~min$^{-1}$ for 100 cycles.
The strain set, or unrecovered strain, is defined as the residual strain and was quantified for each cycle.

Mullins-type nonlinearity and elasticity were probed via step-cyclic loading.
During the loading-unloading cycles, specimens were sequentially strained to a maximum applied strain of 10\%--800\% at a rate of 50~mm~min$^{-1}$.


\subsection{Fracture Energy}
\label{subsec:fracture-energy}

Fracture energy was measured using unnotched and notched samples.
The notched samples with a $\sim$2-mm (30\%) central precut were stretched to induce crack propagation ($n = 3$).
Tests were conducted with the 50~N load cell at a constant extension rate of 500~mm~min$^{-1}$.

The fracture energy ($\Gamma$) was calculated by the areal integration under the stress-strain curve for the unnotched specimen until $\varepsilon_c$:
\begin{equation}
    \Gamma = L_0 \int_0^{\varepsilon_c} \sigma \, d\varepsilon
    \label{eq:fracture-energy}
\end{equation}

The fractocohesive length ($l_f$) is defined as:
\begin{equation}
    l_f = \frac{\Gamma}{W_f}
    \label{eq:fractocohesive-length}
\end{equation}


\subsection{Puncture Tests}
\label{subsec:puncture-tests}

Puncture tests were performed using a 30~kN load cell.
An 18-gauge sharp cylindrical needle (tip radius $\approx$ 0.2~mm) was driven at 50~mm~min$^{-1}$ through elastomer films 0.2--0.5~mm thick, clamped between concentric circular fixtures (20~mm aperture).
Force--displacement curves were recorded continuously to quantify puncture resistance.


\subsection{Compression of 3D-Printed Scaffolds}
\label{subsec:compression-tests}

Uniaxial compression tests were performed on a 3D-printed gyroid scaffold to assess recovery and densification.
The gyroid was printed from a TMPTA/NRL (42/58~wt\%) DEPR ink with Tartrazine dye.
Tests were conducted at room temperature (50~N load cell, 50~mm~min$^{-1}$) using two protocols.
Each printed part with dimensions of $10.94 \times 10.94 \times 12.94$~mm was tested under two compression protocols:
\begin{enumerate}[label=\alph*)]
    \item \textbf{Cyclic durability}: 1,000 loading--unloading cycles at 20\% compressive strain.
    \item \textbf{Step-recovery}: Step-cyclic compression to maximum strain of 25, 50, 75\% strain, with complete unloading between steps to allow recovery.
    Finally, samples were compressed to densification ($>$86\% strain) to determine the ultimate compressive strength.
\end{enumerate}


\section{DLP 3D Printing}
\label{sec:dlp-printing}

3D printing was performed on a home-made \ac{DLP} 3D printer.
A customized resin vat with an oxygen-permeable window made of Teflon AF-2400 (Biogeneral, Inc., USA) was prepared.
A digital-micromirror-device-based UV projector (DLP4710 1080p, UV-LED; Wintech, USA) with a pixel resolution of 1920 $\times$ 1080 was used as the light source (wavelength: 385~nm).
The projector light intensity was 7.7~mW~cm$^{-2}$, measured by a handheld optical power meter (PM100D; Thorlabs GmbH, USA).

CAD files for the printed part were designed in SolidWorks (Dassault Systèmes, USA) or obtained from online repositories.
The exported STL files were sliced into PNG images using the Creation Workshop software (Wanhao, China).
The degassed emulsion photoresin was added to the vat before printing.
The build platform was elevated by a 150-mm translation stage with a stepper motor and Integrated Controller (LTS150, Thorlabs, USA).
The printing layer thickness was 75~$\mu$m, and the exposure time was 20~s per layer.
After printing, the parts were removed from the build platform and post-treated in an oven at 70°C overnight.


\subsection{Measurement of Curing Depth}
\label{subsec:curing-depth}

The curing depth was quantified using a confined-film method.
Briefly, two microscope glass slides (75~mm $\times$ 25~mm) were separated by two spacers (shims) with a nominal thickness of 1.5~mm, forming a uniform gap.
The emulsion photoresin was dispensed into the gap and spread to obtain a laterally uniform resin layer.
The assembly was then exposed to UV light for a prescribed time under the same wavelength and irradiance conditions used for printing (385~nm; 7.7~mW~cm$^{-2}$).
After exposure, the top glass slide was removed, and uncured resin was gently wiped off.
The thickness of the cured film was measured at three locations using a digital caliper, and the average value was reported as the curing depth.


\subsection{Resolution and Geometric Fidelity Characterization}
\label{subsec:resolution-characterization}

Printing resolution was evaluated using a custom-designed test stage fabricated from ABS.
The emulsion photoresin was evenly coated onto the exposure region of the stage to form a thin, uniform resin layer.
A radial spoke test pattern was projected and exposed for 6~min under the same UV conditions as above.
After exposure, the specimen was gently rinsed with deionized (DI) water to carefully remove uncured resin and avoid mechanical damage to the cured features.
The printed petal geometry was then imaged, and the petal opening angle of individual petals was measured and compared with the corresponding digital model.
The ratio between the angular size ($\theta$) of the printed part and the CAD design ($\theta_0$) was used as a metric to quantify geometric fidelity.


\subsection{Dip-Coating Fabrication and Pneumatic Inflation}
\label{subsec:dip-coating}

To demonstrate the material's processing versatility, complex geometries were fabricated via dip-coating.
An industrial-grade dip mold (ceramic or aluminum) was immersed in the liquid dual emulsion for 5~s.
Upon removal, the coated layer was cured under UV light (30~mW$\cdot$cm$^{-2}$).
To build sufficient wall thickness for handling, this dip-cure cycle was repeated four additional times (five layers total).
Finally, the multilayered sample was dehydrated at 70°C overnight to promote latex coagulation and film formation.

The toughness and flexibility of the dip-coated samples were qualitatively assessed via a pneumatic inflation test.
A cured, dip-coated balloon specimen was connected to a compressed-air line within a fume hood.
A standard pipette discharge tip was utilized as a capillary adaptor to interface the sample with the air supply.
The sample was successfully inflated using compressed air, demonstrating the material's ability to undergo significant deformation without rupture.


\subsection{High-Volume--Low-Pressure Spray-Coating and Hydrophobicity Demonstration}
\label{subsec:hvlp-spray}

To demonstrate the processability and versatility of the dual-emulsion, a high-volume, low-pressure (HVLP) spray-coating protocol was applied to diverse substrates, including silicone elastomers, polyethylene discs, polycarbonate films, and a porous almond cake model.
Prior to application, the formulation (containing 25~wt\% photoresin and 0.05~wt\% Tartrazine dye) was verified to have a viscosity $<$40~DIN-seconds, allowing direct atomization without solvent thinning using a Slikwave CN-7000 sprayer fitted with a 1.2~mm nozzle.
Spray coating was performed using an electric HVLP paint sprayer (Suzhou ChengZi, China).
The emulsion was sprayed onto the substrates using a horizontal fan pattern at a working distance of 15--20~cm and a traverse speed of 10~cm~s$^{-1}$ with 50\% overlap, followed by ambient drying for 30~minutes and UV curing for 10~minutes to crosslink the TMPTA phase.
The robustness of the resulting hydrophobic barrier was validated via an immersion stress test, where a coated almond cake subjected to stirring at $\sim$400~rpm in deionized water retained its structural integrity and yellow coloration after 4.5~minutes, whereas the uncoated control disintegrated; this confirmed the coating's ability to provide water resistance and dye retention across materials with varying surface energies and porosities.


\subsection{Gel Permeation Chromatography (GPC)}
\label{subsec:gpc}

The molecular weight distribution of the purified natural rubber latex was characterized using a Viscotek GPCmax system (Malvern Panalytical, UK) equipped with a Model 302-050 tetra-detector array (RI, UV, differential viscometer, and LALS).
Separation was performed using two mixed-porosity PolyPore columns (5~$\mu$m particle size) in series, maintained at 40°C with tetrahydrofuran (THF) as the mobile phase at a flow rate of 1.00~mL~min$^{-1}$.

Prior to analysis, the dried rubber sample was dissolved in inhibitor-free THF under stirring for 40 days to ensure complete dissolution, and the solution was subsequently filtered through a 0.4~$\mu$m PTFE syringe filter.
Absolute molar masses were calculated via universal calibration (Omnisec software), yielding:
\begin{itemize}
    \item Number-average molecular weight ($M_n$): $1.02 \times 10^6$~g~mol$^{-1}$
    \item Weight-average molecular weight ($M_w$): $2.13 \times 10^6$~g~mol$^{-1}$
    \item Polydispersity index (\ac{PDI}): 2.09
\end{itemize}
