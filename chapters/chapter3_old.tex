%!TEX root = ../dissertation.tex
%%%%%%%%%%%%%%%%%%%%%%%%%%%%%%%%%%%%%%%%%%%%%%%%%%%%%%%%%%%%%%%%%%%%%%%%%%%%%%%%
% chapter3.tex: Research Methodology
%%%%%%%%%%%%%%%%%%%%%%%%%%%%%%%%%%%%%%%%%%%%%%%%%%%%%%%%%%%%%%%%%%%%%%%%%%%%%%%%
\chapter{Research Methodology}
\label{ch:methodology}


\section{Sourcing, Traceability, and Rationale: Natural Rubber Latex Preservation Systems}
\label{sec:sourcing}


\subsection{Overview of Preservation Methods as the Master Variable}
\label{subsec:preservation-overview}

The fundamental research design rests upon systematic comparison of two distinct preservation methodologies for \ac{NRL}, each representing different approaches to maintaining polymer stability and functional properties.
This comparison serves as the Master Variable throughout the experimental program, directly addressing the research objective to define how preservation chemistry alters \ac{NRL} microstructure and flow characteristics.
The primary motivation for this comparison derives from documented preservation and processing challenges in \ac{NRL} production, particularly concerning ammonia toxicity, environmental sustainability, and the need to maintain microbial stability without hazardous volatiles.


\subsection{Ammoniated Latex System}
\label{subsec:ammoniated-system}

AFLatex Technologies LDA (Victoria, Caldas, Colombia) provided the ammoniated \ac{NRL} sample, representing the historical preservation standard.
This ammoniated latex was supplied as a centrifuged system with 60\% \ac{DRC} according to ASTM D1076-15 classification standards.
The ammoniated formulation employs ammonia as the primary preservative and stabilizer, which has been the traditional approach for \ac{NRL} conservation for decades.
However, this material serves a dual role in the research: it functions both as a baseline against which eco-preserved systems are evaluated and as a means to investigate ammonia's documented effects on protein and phospholipid retention within the latex colloidal system.
Due to traditional production practices and transportation constraints, acquiring an equivalent ammoniated latex serum sample for direct compositional comparison proved infeasible; consequently, the 60\% \ac{DRC} ammoniated latex represents the primary ammoniated reference material in this study.


\subsection{Eco-preserved Latex Systems}
\label{subsec:eco-preserved}

AFLatex Technologies LDA provided two distinct eco-preserved \ac{NRL} formulations, both ammonia-free, which form the core of the comparative preservation study.
The first system, designated Alfa, employs a preservation chemistry combining ethoxylated tridecyl alcohol and hydrofluoric acid.
The second system, designated Beta, utilizes linear dodecylbenzene sulfonic acid as the primary preservative.
Both materials were supplied in two distinct solid content formulations, approximately 60\% and 30\% \ac{DRC}, allowing investigation across a range of particle volume fractions while maintaining identical preservation chemistry.
Additional latex serum was provided to prepare intermediate concentrations as needed.
This dual-formulation approach enables systematic evaluation of how concentration influences the preservation-dependent microstructure and rheological properties.


\subsection{Reference and Synthetic Polymer Materials}
\label{subsec:reference-materials}

To contextualize the natural rubber latex system within the broader landscape of elastomeric polymers, two reference materials were included.
Synthetic polyisoprene (L-IR-50) with an average molecular weight of 54,000~Da was donated by Kuraray Co. Ltd. (Tokyo, Japan).
Deproteinized and saponified liquid natural rubber (DPR-40) with an average molecular weight of 40,000~Da was donated by DPR Industries, a division of Pacer Industries Inc. (Coatesville, PA, USA).
These materials allow distinction between the rheological and NMR signatures attributable to preservation chemistry versus those arising from the native protein-lipid matrix inherent to natural rubber latex.
By comparing native \ac{NRL} (ammoniated and eco-preserved) against synthetic polyisoprene and deproteinized rubber, the study isolates the specific effects of preservation-dependent protein and phospholipid behavior on macroscopic properties.


\subsection{Pre-receipt Specifications}
\label{subsec:pre-receipt}

The receipt-stage characterization was used to benchmark incoming natural rubber latex against ISO 2004:2010 and related procedures (e.g., ASTM D1076 for \ac{DRC}).
The objective is lot-level traceability and verification of baseline quality prior to formulation and rheological testing.
Key standardized indicators include mechanical stability (\ac{MST}), volatile fatty acid number/index (\ac{VFA}), solids content (\ac{TSC}/\ac{DRC}), and Brookfield viscosity.
Table~A1 reports measured values (average $\pm$ standard deviation) for the latex lots used in this work, alongside any supplier \ac{CoA} values when available.
Table~A2 lists the corresponding ISO specification limits by latex type (centrifuged/creamed; HA/LA/XA).
Together, these tables link experimental rheological findings to standardized quality metrics and clarify preservative- and processing-dependent differences relevant to performance.


\subsection{Incoming Verification}
\label{subsec:incoming-verification}

Upon receipt, each latex batch was independently verified to confirm it matched the supplier's \ac{CoA} and met baseline quality requirements.
All results were recorded on an Incoming Verification Sheet linked to supplier, lot/batch code, ship/storage conditions, and the intended use condition; batches failing criteria were quarantined or rejected.

\ac{DRC} was measured by ASTM D1076: ${\sim}10$~g latex was diluted to ${\sim}25$~wt\% total solids, coagulated with 2~wt\% acetic acid under stirring, washed/rolled, then dried at 70~°C (or 55~°C if oxidation was observed) to constant mass.
Batches were accepted only when measured \ac{DRC} agreed with the supplier value within $\pm 1$~wt\% (absolute); out-of-spec results were repeated once, and persistent deviation triggered quarantine/rejection.
pH and conductivity were measured at 23~°C using a calibrated pH electrode (ASTM E70 practice) and a conductivity probe, respectively, and recorded without adjustment during verification.

Particle-level stability was checked by \ac{DLS}/zeta potential using a 1:100--1:1000 dilution in a defined ionic medium (10~mM electrolyte) to reduce double-layer artifacts and multiple scattering; Z-average diameter, \ac{PDI}, and zeta potential were reported with dilution factor and diluent composition.
Additional receipt screening included appearance (color, odor, phase separation) and visible coagulum assessed by sieving through a 100~$\mu$m mesh; Brookfield viscosity was used as an optional sanity check against internal/supplier baselines, and an aerobic plate count was performed when bioburden information was required (otherwise recorded as ``not evaluated'').


\subsection{Spectroscopic and Analytical Solvents}
\label{subsec:solvents}

Deuterium oxide (D$_2$O, CAS: 7789-20-0) and deuterated chloroform (CDCl$_3$, CAS: 865-49-6, 99.8 atom\% deuteration, containing 0.03\% tetramethylsilane [\ac{TMS}] as internal standard) were purchased from Merck (Sigma-Aldrich).
Both solvents were stored over 4~\AA\ molecular sieves to maintain isotopic purity and prevent isotopic exchange with atmospheric moisture.
These deuterated solvents are essential for $^1$H and $^{13}$C NMR spectroscopy, enabling direct observation of polymer chain dynamics, protein interactions, and any preservation-dependent chemical shifts or relaxation behavior that would remain invisible in conventional hydrogenated solvents.


\subsection{Photopolymerization and Surfactant Reagents}
\label{subsec:reagents}

\ac{TMPTA} (technical grade, 246808), phenylbis(2,4,6-trimethylbenzoyl)-phosphine oxide (\ac{TPO}-L, 97\%, 511447), tartrazine (Acid Yellow 23, $\geq$~85\%, T0388), and Tween 20 (polyoxyethylene (20) sorbitan monolaurate) were obtained from Millipore Sigma, USA.
Ebecryl 114 (2-phenoxyethyl acrylate, 024572A), Ebecryl IBOA (isobornyl acrylate monomer, 024944B01Z01), and Ebecryl 8413 (pigment-grind urethane-acrylate oligomer, 029588A) were supplied by Allnex, USA.
\ac{HDDA} (99\%, 043203.30) was purchased from Thermo Scientific, USA.
Span 80 (sorbitan monooleate, S0060) was ordered from TCI Chemicals.
Acetate buffer solution (pH 4.66, 1.07827.1000) was acquired from Merck, Germany.


\section{Rheological Characterization}
\label{sec:rheology}

Understanding how \ac{NRL} flows and deforms under stress is critical because flow behavior reflects the microstructure and dictates processability in additive manufacturing.
This section details the rheological protocols; the colloidal phenomena being probed and the models used to interpret the data.


\subsection{Measurement Types and Protocols}
\label{subsec:measurement-protocols}

\subsubsection{Sample Preparation and Volume-Fraction Series}
\label{subsubsec:sample-prep}

Rheology was performed on latex suspensions prepared over a solids volume-fraction range of $\phi = 0.2$--$0.6$ to isolate the effect of concentration on flow and viscoelastic response.
For the ammonia-free latex systems, the stock latex at the highest concentration was diluted to target $\phi$ using the matching latex serum to preserve the native ionic environment and minimize artifacts in colloidal interactions during dilution.
For ammonia-preserved latex, dilution was performed using deionized water rather than ammonia serum due to handling hazards; pH and conductivity were recorded for each diluted sample to document any shift in ionic strength introduced by this approach.
Target $\phi$ values were calculated from measured \ac{DRC}/\ac{TSC} and the known mass fractions used in each dilution.


\subsubsection{Rotational Rheometry (Parallel Plates)}
\label{subsubsec:rotational-rheometry}

Steady and oscillatory measurements were conducted using a NETZSCH Kinexus rotational rheometer with a 40~mm upper / 60~mm lower parallel-plate configuration and a 0.5~mm gap at room temperature.
Approximately 1~mL of latex was loaded for each test.

\begin{enumerate}[label=(\roman*)]
    \item \textbf{Steady shear ramps} were performed from 0.01~s$^{-1}$ to 300~s$^{-1}$ to obtain viscosity--shear rate curves and identify Newtonian plateaus and shear-thinning regions. Where thixotropy was relevant, an up--down ramp (0.01 $\rightarrow$ 300 $\rightarrow$ 0.01~s$^{-1}$) was used to assess hysteresis between the increasing and decreasing branches.
    
    \item \textbf{Oscillatory amplitude sweeps} were used to determine the linear viscoelastic window and quantify storage and loss moduli ($G'$, $G''$) as a function of strain amplitude. Sweeps were collected over 0.01--200\% strain at fixed frequencies of 0.1~Hz, 1~Hz, and 10~Hz for each $\phi$.
    
    \item \textbf{Shear start/recovery (thixotropy) tests} were conducted using a three-interval thixotropy test (3ITT): a low-shear interval to establish a reference state, a high-shear interval to induce structural breakdown, and a final low-shear interval to quantify recovery. The imposed shear rates were selected within the same bounds as the steady ramps (low shear near 0.01~s$^{-1}$, high shear near 300~s$^{-1}$), and recovery was evaluated by comparing the post-shear viscosity to the initial reference value.
\end{enumerate}


\subsubsection{Computational Fluid Dynamics: Taylor--Couette Measurements}
\label{subsubsec:cfd}

At higher volume fractions where parallel-plate testing can be affected by wall slip and particle migration, a Taylor--Couette (concentric cylinder) geometry was used to provide more reliable high-$\phi$ measurements.
The inner cylinder radius was 0.64~cm, and the outer cylinder radius was 2.54~cm; the inner cylinder rotated at a fixed 55~rpm while the outer cylinder remained stationary.

To interpret concentration nonuniformity and migration trends observed in Couette flow, a simplified two-phase (suspension balance) framework was used: the mixture flow field is solved with no-slip at the walls, and the particle phase is allowed to redistribute via a concentration-transport equation that captures shear-induced migration and buoyancy/settling effects.
This modeling was used as an interpretive tool to confirm that observed viscosity changes with $\phi$ are consistent with expected migration/stability behavior in Couette flow and to support attributing trends to intrinsic particle/serum effects rather than measurement artifacts.


\subsection{Rheology Interpretation Framework}
\label{subsec:rheology-framework}

To interpret viscosity--shear rate data across volume fraction $\phi$, a microstructure-based picture is used in which latex particles form transient linkages (flocs/bridges) at rest and under low shear, and these linkages are progressively disrupted under increasing shear.

The internal structural state is represented by $N$, the instantaneous number (or density) of effective interparticle linkages contributing to resistance to flow, and $N_0$, the maximum/initial linkage density at rest (i.e., the ``fully structured'' reference state).
Under shear, linkages break at a rate that scales with both how much structure exists and how strong the imposed deformation is.
This is written as a breakage term $k_d N \dot{\gamma}^m$, where $k_d$ is the breakage rate constant, $\dot{\gamma}$ is shear rate, and $m$ is the shear-sensitivity exponent (higher $m$ means structure is more sensitive to shear).
When shear is reduced or removed, linkages reform by Brownian-driven encounters and attractive interactions; reformation is modeled as $k_r(N_0 - N)$, where $k_r$ is the reformation rate constant and $(N_0 - N)$ is the ``remaining capacity'' for rebuilding structure.
Combining these gives:
\begin{equation}
    \frac{dN}{dt} = -k_d N \dot{\gamma}^m + k_r(N_0 - N)
    \label{eq:linkage-kinetics}
\end{equation}

At steady state ($dN/dt = 0$), the normalized structure becomes:
\begin{equation}
    \frac{N}{N_0} = \frac{1}{1 + a\dot{\gamma}^m}, \quad a = \frac{k_d}{k_r}
    \label{eq:steady-state-structure}
\end{equation}

Here $a$ is a dimensionless ratio of timescales: large $a$ corresponds to fast breakdown relative to recovery (or slow recovery relative to breakdown), and small $a$ corresponds to structure that reforms rapidly compared to the imposed breakdown.

The link between microstructure and macroscopic flow is made by assuming that viscosity increases with the fraction of linkages remaining.
The Cross-form expression uses two limiting viscosities: $\eta_0$, the zero-shear viscosity (low shear, structure close to intact, $N \approx N_0$), and $\eta_\infty$, the infinite-shear viscosity (high shear, structure largely disrupted, $N \rightarrow 0$).
The steady shear viscosity is then written as:
\begin{equation}
    \eta(\dot{\gamma}) = \eta_\infty + \frac{\eta_0 - \eta_\infty}{1 + a\dot{\gamma}^m}
    \label{eq:cross-model}
\end{equation}

In practice, $\eta_0$, $\eta_\infty$, $a$, and $m$ are extracted by fitting each $\eta(\dot{\gamma})$ curve at fixed $\phi$.
Tracking these fitted parameters versus $\phi$ provides a compact way to quantify how concentration and preservation chemistry change (i) the strength of the structured network (via $\eta_0$), (ii) the ``fully broken'' baseline (via $\eta_\infty$), and (iii) the shear sensitivity and kinetic balance of breakdown vs recovery (via $m$ and $a = k_d/k_r$).

Time dependence (thixotropy) is interpreted with the same variables by comparing the reformation timescale $1/k_r$ to the experimental observation window.
When $1/k_r$ is short relative to the test duration, viscosity recovers during a down-ramp or post-shear interval because $N$ can return toward $N_0$.
In contrast, when recovery is not observed, two limiting interpretations are used: (1) recovery is intrinsically slow ($k_r$ small, so $1/k_r$ is long), or (2) the system is near/above a critical packing state where particle mobility is constrained, so link reformation is effectively suppressed.
Operationally, this second limit is represented as $k_r \rightarrow 0$ at or above an effective critical volume fraction $\phi_c$, implying $N/N_0 \rightarrow 0$ over accessible times and $\eta(\dot{\gamma}) \rightarrow \eta_\infty$ after shear-induced breakdown.
This provides a mechanistic criterion for identifying $\phi_c$: the onset of persistent hysteresis (non-recovering viscosity upon ramp-down or in 3ITT recovery) indicates that the microstructure is no longer reversibly re-forming on experimental timescales, consistent with packing-limited dynamics.


\subsection{Predictive Viscosity Models}
\label{subsec:predictive-viscosity-models}

Predictive viscosity models serve as a bridge connecting microscopic energy states to macroscopic flow behavior.
The historical foundation is Einstein's 1905 relationship for infinitely dilute, hard spheres.
However, as particle concentration increases, particles begin to interact.
The Batchelor extension accounts for these pairwise hydrodynamic interactions and the thermodynamic restoring forces required to maintain the Brownian microstructure.
As the volume fraction approaches maximum packing ($\phi_m$), many-body constraints dominate, requiring divergence models like Krieger--Dougherty to capture the rapid rise in viscosity due to ``caging'' and loss of free volume.

The historical starting point is Einstein's 1905 result for infinitely dilute, rigid, noninteracting spheres in a Newtonian solvent.
Under the assumptions of perfect no-slip at the particle surface, spherical geometry, and negligible particle--particle interactions, the relative viscosity is:
\begin{equation}
    \eta_r \equiv \frac{\eta}{\eta_s} = 1 + [\eta]\phi = 1 + 2.5\phi
    \label{eq:einstein-viscosity}
\end{equation}

Here $[\eta] = 2.5$ is the intrinsic viscosity of a sphere, obtained from solving the Stokes (creeping-flow) problem around an isolated particle.
In practice, $\phi$ is the key input parameter; particle size $a$ is typically measured via SEM/TEM/\ac{DLS} but note that in the strict Einstein limit the viscosity depends on volume fraction, not size, because the disturbance flow scales self-similarly with $a$ at fixed $\phi$.

Moving beyond ``infinitely dilute'' means admitting that particles feel each other.
The next correction is the $\phi^2$ term: once pairs exist, the flow disturbance from one particle modifies the stress on another.
Early extensions treat how particle concentration, size, and dispersion quality influence viscosity by adding higher-order terms:
\begin{equation}
    \eta_r = 1 + 2.5\phi + k_2\phi^2 + \cdots
    \label{eq:virial-expansion}
\end{equation}
where $k_2$ encodes pairwise hydrodynamic interactions plus any microstructural bias (aggregation, anisotropy, etc.).
This is also where real-world complications show up: modest aggregation increases dissipation because clusters behave like effectively larger objects and generate stronger local velocity gradients; deviations from sphericity introduce rotational dynamics and orientation distributions; and polydispersity shifts packing and crowding behavior, often requiring an ``effective packing'' description rather than a single $\phi$.

Batchelor's contribution is useful to separate two ideas that often get mixed: hydrodynamic interactions versus Brownian (thermodynamic) microstructure.
In a semidilute but still nonaggregating suspension, even if particles are ``noninteracting'' in the sense of having no electrostatic/van der Waals potential, the solvent still couples them hydrodynamically.
Batchelor-type hydrodynamic analyses add the pairwise Stokesian interactions, yielding a corrected $\phi^2$ coefficient (often quoted around $k_2 \sim 5$ for the purely hydrodynamic piece in the dilute expansion).
Conceptually, this term is ``how much extra stress comes from forcing solvent to squeeze and circulate around neighboring particles,'' even when the equilibrium structure remains isotropic.

Batchelor then extended the framework to Brownian hard spheres at low shear, where the microstructure is not arbitrary; it is maintained by thermal motion.
Under small, applied shear, the equilibrium isotropic $g(r)$ is slightly distorted.
The key point is that the stress is no longer purely hydrodynamic: it also includes an additional contribution from the thermodynamic restoring force associated with distorting the Brownian microstructure.
In dilute form, this appears as a small modification of the $\phi^2$ term (often expressed as a hydrodynamic part plus a Brownian part), reflecting that viscosity at low Pe contains both (i) dissipation from flow around particles and (ii) stress from maintaining/relaxing the Brownian structure.
This is the bridge from ``pair interactions'' to ``microstructure as a state variable,'' even before you reach high $\phi$---structure matters because Brownian motion continuously rebuilds it.

Once $\phi$ becomes large enough, however, pairwise corrections stop being the main story.
Many-body constraints appear, particles become caged by neighbors, the free volume collapses, and viscosity rises dramatically as the system approaches a packing-limited state.
That shift motivates semi-empirical ``crowding'' models that effectively sum higher-order interactions into a closed form.
Mooney's model is an early and useful example:
\begin{equation}
    \eta_r = \exp\left(\frac{[\eta]\phi}{1 - \phi/\phi_m}\right)
    \label{eq:mooney}
\end{equation}
where $\phi_m$ is a maximum-packing (or jamming) parameter capturing the onset of strong crowding.
The interpretation is physical and microstructural: as $\phi \rightarrow \phi_m$, the denominator shrinks, indicating that rearrangements require increasingly cooperative motion, so stress rises superlinearly.

Krieger--Dougherty provides a widely used refinement that makes the divergence explicit while retaining the intrinsic viscosity:
\begin{equation}
    \eta_r = \left(1 - \frac{\phi}{\phi_m}\right)^{-[\eta]\phi_m}
    \label{eq:krieger-dougherty}
\end{equation}

This form is especially practical for high-solids formulations (pastes, ceramic slurries, concentrated latexes) because it cleanly separates what you often know ($[\eta] \approx 2.5$ for near-spheres) from what you must fit ($\phi_m$, which depends on size distribution, softness, shape, and dispersion quality).
In microstructural language: $\phi_m$ is a proxy for ``how efficiently this system can pack before it dynamically arrests.''

It helps to be explicit about what changes after phase separation or flocculation; the microstructure stops being ``crowded hard spheres'' and becomes a load-bearing network.
That network introduces a yield stress because at low stress the structure does not continuously rearrange---it resists like a weak solid.
The simplest yield-stress constitutive model is Bingham:
\begin{equation}
    \tau = \tau_y + \eta_p\dot{\gamma}
    \label{eq:bingham}
\end{equation}
but for colloidal gels and flocculated suspensions the more flexible---and usually more accurate---choice is the Herschel--Bulkley form:
\begin{equation}
    \tau = \tau_y + K\dot{\gamma}^n \quad (0 < n < 1 \text{ for shear thinning})
    \label{eq:herschel-bulkley}
\end{equation}

This single equation is valuable because it connects the existence of a microstructural network ($\tau_y$) with its flow-induced breakdown and restructuring ($K$, $n$).
In other words: $\tau_y$ encodes ``how strong is the network at rest,'' while $n$ encodes ``how rapidly does the structure yield and thin under flow.''


\subsection{Addressing Measurement Challenges and Micro-Level Effects}
\label{subsec:measurement-challenges}

A rotational rheometer does not measure viscosity directly; it measures torque $M$ and angular velocity $\Omega$, from which shear stress $\tau$, shear rate $\dot{\gamma}$, and apparent viscosity $\eta_{\text{app}}$ are inferred via geometry-dependent factors $K_\tau$ and $K_\gamma$ obtained by solving the Navier--Stokes equations under idealized conditions.
In compact form:
\begin{equation}
    \tau = K_\tau M, \quad \dot{\gamma} = K_\gamma \Omega, \quad \eta_{\text{app}} = \frac{\tau}{\dot{\gamma}} = \frac{K_\tau}{K_\gamma} \frac{M}{\Omega}
    \label{eq:rheometer-relations}
\end{equation}

These relations are only valid if three physical assumptions hold:
\begin{enumerate}[label=(\roman*)]
    \item no slip at the walls ($v_{\text{fluid}} = v_{\text{wall}}$),
    \item homogeneity across the gap ($\phi$ and hence $\eta$ independent of position),
    \item laminar simple shear without secondary flows.
\end{enumerate}

In concentrated \ac{NRL}, all three are easily violated.
Depletion layers at smooth tools create wall slip, so only a thin solvent-rich layer is actually sheared; the instrument overestimates $\dot{\gamma}$ and underestimates $\eta$.
Shear-induced migration in geometries with strong shear-rate gradients (parallel plate, wide-gap Couette) drives particles from high-shear to low-shear regions, generating a spatially varying $\phi(\mathbf{r})$ and a torque that is a nontrivial average over a heterogeneous microstructure.
At higher rotational speeds in cylindrical geometries, inertial instabilities (Taylor vortices, wavy vortices) add an extra, non-constitutive contribution to the torque, which appears as artificial shear thickening if interpreted with the simple $\eta \propto M/\Omega$ relation.

Mitigation, therefore, combines experimental design and analytical correction to keep the inferred viscosity as close as possible to the true constitutive response.
Wall slip is minimized by using serrated or sand-blasted tools to mechanically couple the bulk to the walls; when unavoidable slip remains, measurements across multiple gaps are analyzed using Mooney-type constructions to estimate slip velocity and correct the true shear rate.
Shear-induced migration is reduced at the source by choosing geometries with nearly uniform shear (small-angle cone-and-plate, narrow-gap Couette) and by limiting measurement times at high shear so that strong concentration gradients do not fully develop.
When gradients are expected, the data are interpreted within a suspension-balance or two-phase framework, where particle fluxes driven by $\nabla\dot{\gamma}$ and $\nabla\phi$ are coupled to a local viscosity $\eta(\phi)$ to rationalize deviations from ideal behavior.
Finally, flow instabilities are avoided by operating below the critical Taylor/Reynolds numbers for the chosen gap-to-radius ratio, and by favoring narrow-gap or outer-rotating Couette configurations that delay the onset of vortices.
Together, these strategies ensure that the reported \ac{NRL} viscosities and fitted constitutive parameters reflect microstructure-controlled material properties rather than artifacts of migration, slip, or inertial flow.


\section{Nuclear Magnetic Resonance (NMR)}
\label{sec:nmr}


\subsection{Overview and General Conditions}
\label{subsec:nmr-overview}

All NMR experiments were performed at 303~K on two solution-state instruments:
\begin{enumerate}[label=(\roman*)]
    \item a Bruker Avance I 600~MHz equipped with a Prodigy cryoprobe (Z150313\_001; cpT4600ss3H\&F-LIN-D-05Z)
    \item a Bruker Nyx / Bruker NEO 500~MHz fitted with a Prodigy-BBO probe (Z130036\_0001; CPP BBO 500S2BB-H\&F-D052LT)
\end{enumerate}

Chemical shifts were internally referenced using \ac{TMS} ($\delta_H = 0$~ppm), CDCl$_3$ ($\delta_H = 7.26$~ppm), or D$_2$O/HDO ($\delta_H = 4.79$~ppm), selected based on the solvent system used for each measurement.


\subsection{Sample Preparation (Solution-State NMR)}
\label{subsec:nmr-sample-prep}

Unless otherwise specified, samples were prepared by dissolving natural rubber latex (\ac{NRL}) at 10~mg per 0.6~mL D$_2$O in standard 5~mm NMR tubes.
Shimming routine (topshim/manual), lock nucleus, number of dummy scans (DS), and receiver gain were fixed across samples.
To capture compositional and processing variability, solution-state NMR was conducted on multiple \ac{NRL}-derived sample classes under both ammonia-preserved and ammonia-free conditions:
\begin{itemize}
    \item field latex (ammonia and ammonia-free),
    \item an industrial serum fraction from the ammonia-free system,
    \item concentrated latex (ammonia and ammonia-free), and
    \item ultracentrifugation-fractionated latex prepared from each preservation condition.
\end{itemize}

\ac{NRL} was fractionated by ultracentrifugation at 25,000~rpm and 277~K, yielding a reproducible three-layer separation: a top cream layer (enriched in intact rubber particles), followed by two aqueous serum layers (Serum B and Serum C).
Serum C exhibited weaker but still recognizable signals relative to Serum B and the cream fraction.
Each fraction (cream, Serum B, Serum C), along with the unfractionated starting material and other sample classes above, was analyzed using consistent 1D and 2D NMR workflows for both ammonia and ammonia-free systems.


\subsection{High-Resolution Solution-State NMR (1D and 2D)}
\label{subsec:high-res-nmr}

The following Bruker pulse programs were used:
\begin{itemize}
    \item $^1$H 1D: zg30 (NS = 8)
    \item $^{13}$C 1D: zgpg30 (NS = 6400)
    \item $^1$H--$^{13}$C \ac{HSQC} (multiplicity-edited): hsqcedetgpsisp2p.3 (NS = 8)
    \item \ac{HMBC}: hmbcgpndqf (NS = 8)
    \item $^1$H--$^1$H \ac{COSY}: cosygpprqf.uw (NS = 64)
    \item $^{31}$P 1D: zgig30 (NS = 256)
    \item Heteronuclear HMBC: HMBC\_Hx.uw (NS = 2)
    \item Multiplicity-edited HSQC (alt): HSQCetfd.uw (NS = 2)
    \item APT: jmod (NS = 128)
    \item DEPT-135: dept-135 (NS = 1005)
\end{itemize}

Data were processed in MestReNova.
Free induction decays (FIDs) were Fourier transformed after zero-filling to 2$\times$ points.
An exponential apodization was applied with line broadening defined as 1/AQ, followed by manual phase correction and baseline correction using a 6th-order polynomial over the full spectral width.
Peak picking, integration ranges, coupling extraction, and cross-peak assignment criteria were applied consistently across conditions.


\subsection{Diffusion-Ordered Spectroscopy (DOSY)}
\label{subsec:dosy}

\ac{DOSY} experiments used the ledbpgp2s sequence with: NS = 16, receiver gain = 3.5, relaxation delay = 2~s, pulse width = 7.07~$\mu$s, and acquisition time = 2.7739~s.
DOSY processing was performed in MestReNova using the Bayesian DOSY Transform.
Key settings included an exponential decay model where diffusion coefficients $D$ were obtained by fitting the diffusion-dependent signal attenuation to the Stejskal--Tanner relation:
\begin{equation}
    I(g) = I_0 \exp\left[-b(g) D\right]
    \label{eq:stejskal-tanner}
\end{equation}
with
\begin{equation}
    b(g) = \gamma^2 g^2 \delta^2 \left(\Delta - \frac{\delta}{3}\right)
    \label{eq:b-factor}
\end{equation}
where $I(g)$ is peak intensity at gradient amplitude $g$, $I_0$ is intensity at $g = 0$, $\gamma$ is gyromagnetic ratio, $\delta$ is gradient pulse duration, and $\Delta$ is diffusion delay.

Hydrodynamic radius was estimated using the Stokes--Einstein equation:
\begin{equation}
    r_h = \frac{k_B T}{6\pi\eta D}
    \label{eq:hydrodynamic-radius}
\end{equation}
where $k_B$ is the Boltzmann constant, $T$ is absolute temperature, and $\eta$ is solvent viscosity.

Autocorrected peak positions were enabled, Repetitions = 1, resolution factor = 1, and 128 spectral points, with logarithmic diffusion axis and autoscaling enabled.


\subsection{Time-Domain NMR Relaxometry (TD-NMR)}
\label{subsec:td-nmr}

TD-NMR was used to probe relaxation dynamics associated with distinct molecular environments.
$T_1$ was measured using an inversion recovery sequence (t1ir, NS = 2) over 11 delay points.
$T_2$ was measured using a \ac{CPMG} sequence with d1 = 4~s, d20 = 0.001~s, L4 = 2, and NS = 8.
TD-NMR processing followed the same core pipeline as solution-state processing (FT, zero-filling/apodization, phase, and baseline correction).
Integrals were computed in MestReNova to generate intensity vs. delay-time curves, exported to Origin, normalized, and fit using an exponential recovery model for $T_1$.


\subsection{Quantification (Models Used)}
\label{subsec:nmr-quantification}

$T_1$ relaxation times were extracted by fitting the inversion-recovery signal intensities to a mono-exponential recovery model with an optional baseline offset:
\begin{equation}
    M(t) = M_0 \left(1 - 2e^{-t/T_1}\right) + C
    \label{eq:t1-recovery}
\end{equation}
where $M(t)$ is integrated signal intensity at the delay time $t$, $M_0$ is the equilibrium magnetization scale factor, and $C$ is an offset term (used if baseline/inversion imperfections require it).

$T_2$ relaxation was quantified from \ac{CPMG} decay curves by fitting integrated echo amplitudes to a biexponential decay model, representing at least two distinct spin populations/environments:
\begin{equation}
    M(t) = A_1 e^{-t/T_{2,1}} + A_2 e^{-t/T_{2,2}} + C
    \label{eq:t2-decay}
\end{equation}
where $A_1$, $A_2$ are component amplitudes (population-weighted contributions), $T_{2,1}$, $T_{2,2}$ are component transverse relaxation times, and $C$ is an optional offset.


\section{Components and Protocols of NRL Photoresin Formulation}
\label{sec:photoresin}

Advanced manufacturing of \ac{NRL} requires a carefully engineered photoresin that balances printability, cure kinetics, and mechanical performance.
This section details the formulation strategy, the materials and equipment used, and the protocols adopted to validate the rheological and photochemical behavior of the \ac{NRL}-based resin.


\subsection{Preparation of the UV-Curable Latex-Scaffold Coalescence}
\label{subsec:uv-curable-latex}

0.9~wt\% \ac{SDS} was added to 60\% \ac{DRC} ammonia-free \ac{NRL} and mixed for 3~min using a magnetic stir bar in a 100~mL beaker.
Subsequently, approximately 4.44~wt\% \ac{HDDA} was incorporated into the solution under low Kelvin lighting conditions with slow dispersion mixing for 5~min.
Following this, approximately 4.5~wt\% \ac{TPO}-L photoinitiator was added, and the mixture underwent a slow 1.5-h mixing process to reduce the operational viscosity further since \ac{NRL} is shear thinning.
The beaker was sealed with Saranwrap to prevent dehydration and excessive air interaction, while an aluminum foil cover was utilized to minimize light exposure.
The low Kelvin lighting was turned off until the mixing was complete.


\subsection{Preparation of Photoresin Emulsion (PRE)}
\label{subsec:pre-preparation}

High-internal-phase oil-in-water (O/W) photoresin emulsions were prepared at a 60:40 oil-to-water weight ratio.
The oil phase (60~wt\% of total emulsion) consisted of the base monomer (\ac{HDDA} or \ac{TMPTA}) supplemented with 1.0~wt\% photoinitiator (\ac{TPO}-L) and 0.75~wt\% low-\ac{HLB} surfactant (Span 80).
The aqueous phase (40~wt\% of total emulsion) consisted of acetate buffer containing 4.25~wt\% high-\ac{HLB} surfactant (Tween 20).
The total surfactant concentration was fixed at 5~wt\% of the final emulsion, with a Span:Tween weight ratio of 3:17 to achieve an effective HLB $\approx$ 11.

Emulsification was performed using an overhead stirrer equipped with a four-blade pitched-blade turbine (MINISTAR 20, IKA Works, Germany).
The oil and aqueous phases were pre-mixed separately and equilibrated to 20--23~°C.
With the aqueous phase stirred at 800--1,200~rpm, the oil phase was added as a thin, steady stream over 60--90~s to maintain O/W morphology and prevent transient inversion.
Following addition, the emulsion underwent a high-shear polishing step at 1,300--1,500~rpm for 30--60~s to narrow the droplet-size distribution and stabilize microscale droplets.


\subsection{Preparation of Dual Emulsion Photoresin (DEPR)}
\label{subsec:depr-preparation}

DEPRs were prepared by blending the O/W \ac{PRE} with \ac{NRL}.
The \ac{NRL} used was an ammonia-free, high-solid content variety (60~wt\% solids; novel preservative method, patent pending).
To evaluate the effect of photoresin loading, a \ac{DOE} approach was used, varying the \ac{PRE} content to 16, 25, 32, and 44~wt\% relative to the total dual-emulsion mass, with the balance comprising the \ac{NRL} stock.
Both \ac{HDDA}- and \ac{TMPTA}-based emulsions were evaluated at these distinct loading levels.

To ensure colloidal compatibility, the aqueous phase of the \ac{PRE} was formulated with an acetate buffer (pH 4.66), selected to match the pH stability range of the \ac{NRL}.
This pH control prevented destabilization upon mixing, maintaining a consistent zeta potential in the final dual emulsion.
The required masses of \ac{PRE} and \ac{NRL} were weighed into an opaque container and stirred gently (300--500~rpm) for approximately 30~s to achieve a uniform dispersion without disrupting droplet or vesicle integrity.
All handling was performed under aluminum foil protection to prevent premature photoinitiator activation.


\subsection{Indirect Manufacturing of Tensile Testing Specimens for UV-Curable Latex-Scaffold Coalescence}
\label{subsec:indirect-manufacturing}

In the fabrication of dogbone-shaped specimens using UV-curable natural rubber latex (\ac{NRL}), a setup inspired by traditional top-down geometric configuration VAT was utilized.
Initially, injection-molded \ac{ASTM} D412 Die C shapes made from high-density polyethylene were used as molds for thermoforming PET sheets.
After softening the sheets sufficiently, the dogbone shapes were formed using the injection-molded Die C molds.
A light source was prepared using an Omniture S2000 high-pressure mercury light guide, which was positioned 10~mm above the mold.
UV-curable \ac{NRL} was applied layer by layer using a 3~mL syringe, following a bottom thin layer method to minimize air bubbles in the PET mold.
Each layer was cured under UV light for 30~seconds, and samples were prepared for two intensities: 18 and 30~mW/cm$^2$.
This layering process was repeated until five layers were added to each specimen, resulting in precisely fabricated samples for mechanical and viscoelastic characterization.

After the indirect 3D printing process, the specimens underwent several treatments to enhance their mechanical integrity and stability.
First, they were soaked in isopropyl alcohol for 30~minutes to remove any residual uncured resin.
Next, they were exposed to low-intensity UV light for 10~minutes from a Black-Ray UV bench lamp (365~nm, 115~V--60~Hz) with an intensity of approximately 10--15~mW/cm$^2$ to further harden the material.
Finally, the samples were placed in an Isotemp vacuum oven (Model 282A) at 65~°C and 30~mmHg for 10~hours to ensure dehydration.
Throughout this process, the weight loss of the specimens was monitored before and after processing to maintain consistency in the material properties.


\subsection{Fabrication of Jammed Microreinforced Elastomers (JMRE) and Controls}
\label{subsec:jmre-fabrication}

A two-stage curing process (UV irradiation followed by thermal treatment) was employed to transform liquid \ac{DEPR} into solid, jammed, micro-reinforced elastomers (\ac{JMRE}).
This nomenclature reflects the transition from a jammed micro-emulsion state to a reinforced elastomeric composite.

\begin{enumerate}[label=\Roman*.]
    \item \textbf{UV-Curing and specimen molding}: To prepare mechanical test specimens, custom molds were fabricated by casting translucent tin-cure silicone (Smooth-On, Macungie, USA) against 3D-printed masters (\ac{ASTM} D638 Type V dogbone geometry for tensile tests; flat sheets for fracture/puncture tests). The liquid \ac{DEPR} was cast into the silicone molds and exposed to UV irradiation (30~mW~cm$^{-2}$) using an OmniCure S200 Elite system. This step locked the photoresin phase (\ac{PRE}) into a rigid porous scaffold, establishing the green composite structure.
    
    \item \textbf{Thermal treatment and latex coalescence}: Immediately following UV curing, the specimens were demolded and transferred to a dehydrator/vacuum oven. This step removed residual water and induced osmotic destabilization, thereby forcing the close-packed rubber particles to coalesce within the photoresin scaffold. The samples were then thermally cured at 70~°C overnight to ensure complete formation of the latex film. Demolding before thermal treatment was critical to minimize shrinkage-induced stress and prevent warping due to thermal expansion mismatches.
    
    \item \textbf{Preparation of control samples}: Control samples of pure porous photoresin were prepared by UV-curing the \ac{PRE} (25~wt\% stock) under identical conditions. Pure natural rubber (NR) controls were prepared by casting the \ac{NRL} into molds and chemically coagulating the surface with alcohol to induce a weak gel state similar to the jammed \ac{DEPR} precursor. These gelled NR samples were then subjected to the same 70~°C thermal treatment to ensure a comparable thermal history and diffusion profile.
\end{enumerate}


\subsection{Dip-Coating Fabrication and Pneumatic Inflation}
\label{subsec:dip-coating}

To demonstrate the material's processing versatility, complex geometries were fabricated via dip-coating.
An industrial-grade dip mold (ceramic or aluminum) was immersed in the liquid dual emulsion for 5~s.
Upon removal, the coated layer was cured under UV light (30~mW$\cdot$cm$^{-2}$).
To build sufficient wall thickness for handling, this dip-cure cycle was repeated four additional times (five layers total).
Finally, the multilayered sample was dehydrated at 70~°C overnight to promote latex coagulation and film formation.

The toughness and flexibility of the dip-coated samples were qualitatively assessed via a pneumatic inflation test.
A cured, dip-coated balloon specimen was connected to a compressed-air line within a fume hood.
A standard pipette discharge tip was utilized as a capillary adaptor to interface the sample with the air supply.
The sample was successfully inflated using compressed air, demonstrating the material's ability to undergo significant deformation without rupture.


\subsection{DLP 3D Printing}
\label{subsec:dlp-printing}

3D printing was performed on a home-made \ac{DLP} 3D printer.
A customized resin vat with an oxygen-permeable window made of Teflon AF-2400 (Biogeneral, Inc., USA) was prepared.
A digital-micromirror-device-based UV projector (DLP4710 1080p, UV-LED; Wintech, USA) with a pixel resolution of 1920 $\times$ 1080 was used as the light source (wavelength: 385~nm).
The projector light intensity was 7.7~mW~cm$^{-2}$, measured by a handheld optical power meter (PM100D; Thorlabs GmbH, USA).

CAD files for the printed part were designed in SolidWorks (Dassault Syst\`{e}mes, USA) or obtained from online makerworld.
The exported STL files were sliced into PNG images using the Creation Workshop software (Wanhao, China).
The degassed emulsion photoresin was added to the vat before printing.
The build platform was elevated by a 150-mm translation stage with a stepper motor and Integrated Controller (LTS150, Thorlabs, USA).
The printing layer thickness was 75~$\mu$m, and the exposure time was 20~s per layer.
After printing, the parts were removed from the build platform and post-treated in an oven at 70~°C overnight.


\subsection{High-Volume--Low-Pressure Spray-Coating and Hydrophobicity Demonstration}
\label{subsec:hvlp-spray}

To demonstrate the processability and versatility of the dual-emulsion, a high-volume, low-pressure (HVLP) spray-coating protocol was applied to diverse substrates, including silicone elastomers, polyethylene discs, polycarbonate films, and a porous almond cake model.
Prior to application, the formulation (containing 25~wt\% photoresin and 0.05~wt\% Tartrazine dye) was verified to have a viscosity $<$40~DIN-seconds, allowing direct atomization without solvent thinning using a Slikwave CN-7000 sprayer fitted with a 1.2~mm nozzle.
Spray coating was performed using an electric HVLP paint sprayer (Suzhou ChengZi, China).
The emulsion was sprayed onto the substrates using a horizontal fan pattern at a working distance of 15--20~cm and a traverse speed of 10~cm~s$^{-1}$ with 50\% overlap, followed by ambient drying for 30~minutes and UV curing for 10~minutes to crosslink the \ac{TMPTA} phase.
The robustness of the resulting hydrophobic barrier was validated via an immersion stress test, where a coated almond cake subjected to stirring at $\sim$400~rpm in deionized water retained its structural integrity and yellow coloration after 4.5~minutes, whereas the uncoated control disintegrated; this confirmed the coating's ability to provide water resistance and dye retention across materials with varying surface energies and porosities.


\section{Material Characterizations}
\label{sec:characterization}


\subsection{Particle Size and Zeta Potential}
\label{subsec:particle-size}

The droplet size and surface charge of the photoresin emulsions were characterized using \ac{DLS} and electrophoretic light scattering (ELS), respectively, on a Malvern Zetasizer Nano ZSP (Malvern Panalytical, UK).
The instrument was equipped with a 10~mW He--Ne laser (632.8~nm) and operated with non-invasive backscatter (NIBS) optics at a detection angle of 173°.

Samples were prepared by diluting the \ac{NRL} or photoresin emulsions into 2-mM acetate buffer (pH 4.66) to a final concentration of 0.01--0.05~wt\% to suppress multiple scattering.
Crucially, the emulsion aliquot was added to the buffer (never buffer to sample) to minimize osmotic shock and pH drift.
Measurements were performed in disposable folded capillary cells (DTS1070) at 25~°C after a 120~s equilibration period.
Hydrodynamic diameters were calculated from the autocorrelation function using the Stokes--Einstein equation, assuming the viscosity of water and a particle refractive index of 1.52 (characteristic of natural rubber).
Electrophoretic mobility was measured using the M3-PALS technique and converted to zeta potential via the Smoluchowski approximation.
Each reported value represents the average of three independent measurements, with each measurement consisting of $\sim$12 sub-runs with automatic attenuation optimization.


\subsection{(Photo)rheology Characterizations}
\label{subsec:photorheology}

Rheological measurements were performed using a Discovery HR-20 hybrid rheometer (TA Instruments, USA) utilizing a 20~mm parallel-plate geometry.
To account for optical differences, the gap height was set to 0.5~mm for O/W photoresin emulsions.
For DEPRs, the gap was reduced to 0.2~mm to ensure uniform UV intensity across the sample depth, thereby mitigating the significant light scattering (Tyndall effect) caused by the rubber phase.

\begin{enumerate}[label=\Roman*.]
    \item \textbf{Shear viscosity}: The flow behavior of the emulsions was characterized via steady-state shear experiments. Viscosity profiles were obtained by ramping the shear rate from 0.01 to 200~s$^{-1}$ at 25~°C, allowing for the evaluation of shear-thinning behavior and stability under flow.
    
    \item \textbf{In-situ photorheology}: Real-time photopolymerization kinetics were monitored using the UV-Curing Accessory Kit (TA Instruments), which comprises a quartz lower plate, a liquid-light-guide assembly, and a closed-loop shielding system. UV irradiation was supplied by an OmniCure Series 2000 system (Excelitas Technologies, USA) equipped with a 200~W mercury arc lamp (320--500~nm). The incident intensity at the sample surface was calibrated to 30~mW~cm$^{-2}$. Curing profiles were recorded via oscillatory time sweeps at a fixed frequency of 5~Hz and a strain amplitude of 0.3\%. This strain was confirmed to be within the \ac{LVR} via amplitude sweeps, which exhibited Type 1 behavior (linear response up to critical deformation) consistent with concentrated emulsions. The testing protocol consisted of a 120~s pre-shear equilibration, followed by 30~s of UV irradiation, and a final 400--600~s post-exposure monitoring period to observe the evolution of storage ($G'$) and loss ($G''$) moduli.
\end{enumerate}


\subsection{Morphological Characterization (SEM)}
\label{subsec:sem}

Surface and cross-sectional morphologies were examined using a Gemini \ac{SEM} 450 (Zeiss, Germany).
The microscope was operated at an accelerating voltage of 3.00~kV with a working distance of $\sim$9.2~mm to minimize beam damage and charging on the polymeric samples.

\begin{enumerate}[label=\Roman*.]
    \item \textbf{Porous photoresin scaffolds}: Porous PR samples were mounted by pressing aluminum stubs equipped with double-sided conductive carbon tape directly onto the sample surface to preserve the native porous architecture.
    
    \item \textbf{JMRE samples}: To analyze the internal microstructure and failure mechanisms, imaging was performed on the fracture surfaces of specimens recovered after tensile testing.
    
    \item \textbf{Sputter coating}: Before imaging, samples were sputter-coated with a finer-grained Platinum coating to visualize the tiny pores without introducing artificial roughness. With the \ac{JMRE}, given that the fracture surface is rough and macroscopic relative to pores, conventional gold was sufficient for electrical conductivity using a Leica EM ACE600 high-vacuum coater. To mitigate charging effects on the rough fracture surfaces, the coating thickness was optimized to 10~nm for the \ac{JMRE} composites, while a 5~nm layer was applied to the porous PR samples to prevent obscuring fine pore details.
\end{enumerate}


\section{Mechanical Characterizations}
\label{sec:mechanical-testing}

Mechanical testing, including quasi-static tensile testing, cyclic fatigue, step-cyclic loading, fracture energy, and puncture resistance, was performed using a universal testing machine (Instron 5967, USA).
The system was equipped with interchangeable 50~N and 30~kN load cells.
Soft samples were gripped with Instron BioPlus pneumatic grips, while higher-load experiments used wedge-action mechanical grips.


\subsection{Uniaxial Tension Test}
\label{subsec:tensile-test}

Quasi-static tensile tests were performed on Type V dog-bone specimens at a constant crosshead speed of 500~mm~min$^{-1}$ until failure ($n = 4$--5).
Engineering stress ($\sigma$) was calculated as the measured force divided by the initial cross-sectional area, and engineering strain ($\varepsilon$) as the displacement divided by the initial gauge length.
Young's modulus ($E$) was determined from the linear slope of the stress-strain curve in the low-strain region (1--10\%).

Fracture energy density ($W_f$), representing the total energy absorption capacity, was calculated by integrating the area under the stress-strain curve of the unnotched specimens with a maximum strain $\varepsilon_m$:
\begin{equation}
    W_f = \int_0^{\varepsilon_m} \sigma \, d\varepsilon
    \label{eq:fracture-energy-density}
\end{equation}


\subsection{Cyclic and Hysteresis Tests}
\label{subsec:cyclic-tests}

The normal stress ($\sigma$) and strain ($\varepsilon$) are defined as the measured force divided by the initial cross-sectional area and the displacement divided by the initial gauge distance, respectively ($n = 3$).
For each cycle, the Strain Set Ratio was calculated to quantify the permanent deformation relative to the applied strain.

Cyclic tests were conducted under displacement control on Type V specimens.
Samples were cycled at 100\% strain amplitude and a crosshead speed of 50~mm~min$^{-1}$ for 100 cycles.
The strain set, or unrecovered strain, is defined as the residual strain and was quantified for each cycle.

Mullins-type nonlinearity and elasticity were probed via step-cyclic loading.
During the loading-unloading cycles, specimens were sequentially strained to a maximum applied strain of 10\%--800\% at a rate of 50~mm~min$^{-1}$.


\subsection{Fracture Energy}
\label{subsec:fracture-energy}

Fracture energy was measured using unnotched and notched samples.
The notched samples with a $\sim$2-mm (30\%) central precut were stretched to induce crack propagation ($n = 3$).
Tests were conducted with the 50~N load cell at a constant extension rate of 500~mm~min$^{-1}$, and fracture energy was computed from the work of fracture normalized by the fractured surface area.

The critical strain ($\varepsilon_c$) or stretch ratio ($\lambda_c = \varepsilon_c + 1$) was determined from the strain at peak stress in notched samples.
The fracture energy ($\Gamma$) was calculated by the areal integration under the stress-strain curve for the unnotched specimen until $\varepsilon_c$, considering the original sample length $L_0$:
\begin{equation}
    \Gamma = L_0 \int_0^{\varepsilon_c} \sigma \, d\varepsilon
    \label{eq:fracture-energy}
\end{equation}


\subsection{Fractocohesive Lengths}
\label{subsec:fractocohesive}

The ratio of these two parameters, $\Gamma/W_f$, defines a material-specific length called the fractocohesive length ($l_f$), indicating the flaw-sensitive lengths:
\begin{equation}
    l_f = \frac{\Gamma}{W_f}
    \label{eq:fractocohesive-length}
\end{equation}


\subsection{Puncture Tests}
\label{subsec:puncture-tests}

Puncture tests were performed using a 30~kN load cell.
An 18-gauge sharp cylindrical needle (tip radius $\approx$ 0.2~mm) was driven at 50~mm~min$^{-1}$ through elastomer films 0.2--0.5~mm thick, clamped between concentric circular fixtures (20~mm aperture).
Force--displacement curves were recorded continuously to quantify puncture resistance.


\subsection{Compression of 3D-Printed Scaffolds}
\label{subsec:compression-tests}

Uniaxial compression tests were performed on a 3D-printed gyroid scaffold to assess recovery and densification.
The gyroid was printed from a \ac{TMPTA}/\ac{NRL} (42/58~wt\%) \ac{DEPR} ink with Tartrazine dye.
Tests were conducted at room temperature (50~N load cell, 50~mm~min$^{-1}$) using two protocols.
Each printed part with dimensions of $10.94 \times 10.94 \times 12.94$~mm was tested under two compression protocols:
\begin{enumerate}[label=\alph*)]
    \item \textbf{Cyclic durability}: 1,000 loading--unloading cycles at 20\% compressive strain.
    \item \textbf{Step-recovery}: Step-cyclic compression to maximum strain of 25, 50, 75\% strain, with complete unloading between steps to allow recovery. Finally, samples were compressed to densification ($>$86\% strain) to determine the ultimate compressive strength.
\end{enumerate}


\subsection{Measurement of Curing Depth}
\label{subsec:curing-depth}

The curing depth was quantified using a confined-film method.
Briefly, two microscope glass slides (75~mm $\times$ 25~mm) were separated by two spacers (shims) with a nominal thickness of 1.5~mm, forming a uniform gap.
The emulsion photoresin was dispensed into the gap and spread to obtain a laterally uniform resin layer.
The assembly was then exposed to UV light for a prescribed time under the same wavelength and irradiance conditions used for printing (385~nm; 7.7~mW~cm$^{-2}$).
After exposure, the top glass slide was removed, and uncured resin was gently wiped off using Kimtech wipes to avoid damaging the cured layer.
The thickness of the cured film was measured at three locations (e.g., center and two symmetric off-center points) using a digital caliper, and the average value was reported as the curing depth.


\subsection{Resolution and Geometric Fidelity Characterization}
\label{subsec:resolution-characterization}

Printing resolution was evaluated using a custom-designed test stage fabricated from ABS.
The emulsion photoresin was evenly coated onto the exposure region of the stage to form a thin, uniform resin layer.
A radial spoke test pattern was projected and exposed for 6~min under the same UV conditions as above.
After exposure, the specimen was gently rinsed with deionized (DI) water to carefully remove uncured resin and avoid mechanical damage to the cured features.
The printed petal geometry was then imaged, and the petal opening angle of individual petals was measured and compared with the corresponding digital model.
The ratio between the angular size ($\theta$) of the printed part and the CAD design ($\theta_0$) was used as a metric to quantify geometric fidelity.


\subsection{Gel Permeation Chromatography (GPC)}
\label{subsec:gpc}

The molecular weight distribution of the purified natural rubber latex was characterized using a Viscotek GPCmax system (Malvern Panalytical, UK) equipped with a Model 302-050 tetra-detector array (RI, UV, differential viscometer, and LALS).
Separation was performed using two mixed-porosity PolyPore columns (5~$\mu$m particle size) in series, maintained at 40~°C with \ac{THF} supplied directly with the instrument to ensure high purity, serving as the mobile phase at a flow rate of 1.00~mL~min$^{-1}$.

Prior to analysis, the dried rubber sample was dissolved in inhibitor-free \ac{THF} under stirring for 40~days to ensure complete dissolution, and the solution was subsequently filtered through a 0.4~$\mu$m PTFE syringe filter.
Absolute molar masses were calculated via universal calibration (Omnisec software), yielding:
\begin{itemize}
    \item Number-average molecular weight ($M_n$): $1.02 \times 10^6$~g~mol$^{-1}$
    \item Weight-average molecular weight ($M_w$): $2.13 \times 10^6$~g~mol$^{-1}$
    \item Polydispersity index (\ac{PDI}): 2.09
\end{itemize}
